\chapter{El teorema de Jordan--Hölder}

\begin{definition}
\index{Serie de composición}	
\index{Factores!de una serie de composición}
\index{Longitud!de una serie de composición}
Una sucesión de subgrupos $G=G_0\supseteq G_1\supseteq\cdots\supseteq G_r=\{1\}$ de un grupo $G$ es una
\textbf{serie de composición} si cada $G_{i+1}$ es normal en $G_i$ y cada cociente 
$G_i/G_{i+1}$ es simple. En ese caso, el entero $r$ es la \textbf{longitud} de la serie de composición
y los cocientes son los \textbf{factores} de la serie de composición. 
\end{definition}

\begin{example}
$\Sym_5\supseteq\Alt_5\supseteq\{1\}$ es una serie de composición de $\Sym_5$.	
\end{example}

\begin{example}
$\Z$ no admite una serie de composición pues $\Z$ no es simple y cada subgrupo $S$ de $\Z$ cumple que 
$S\simeq n\Z\simeq\Z$.  	
\end{example}

Vimos en el ejemplo anterior que no todo grupo admite una serie de composición. Sin embargo, todo grupo
finito sí lo hará. Esa será nuestra primera observación. 

% todo: ejercicio, N maximal-normal <=> G/N simple

\begin{proposition}
Si $G$ es un grupo finito, entonces $G$ admite una serie de composición.	
\end{proposition}

\begin{proof}
Procederemos por inducción en el orden de $G$. 
Si $G$ es simple, entonces $G\supseteq\{1\}$ es una serie de composición. Si $G$ no es simple, $G$ contiene un subgrupo normal propio $N\ne\{1\}$, que además puede tomarse maximal sobre los subgrupos normales de $G$. La maximal de $N$ entro los normales de $G$ implica que $G/N$ es un grupo simple. Por hipótesis inductiva, $N$ admite una serie de composición
\begin{gather*}
N=N_0\supseteq N_1\supseteq\cdots\supseteq N_r=\{1\}
\shortintertext{y entonces}
G\supseteq N=N_0\supseteq N_1\supseteq\cdots\supseteq N_r=\{1\}
\end{gather*}
es una serie de composición para $G$ pues $G/N$ es simple.
\end{proof}

\begin{definition}
\index{Equivalencia!de series de composición}
Dos series de composición
\begin{align*}
G=G_0\supseteq G_1\supseteq\cdots\supseteq G_r=\{1\}, &&
G=H_0\supseteq H_1\supseteq\cdots\supseteq H_r=\{1\}	
\end{align*}
de un grupo $G$ se dirán \textbf{equivalentes} si existe $\sigma\in\Sym_r$ tal que 
\[
G_{i-1}/G_i\simeq H_{\sigma(i)-1}/H_{\sigma(i)}
\]
para todo $i\in\{1,\dots,n\}$.  
\end{definition}

\begin{example}
Sea $G=\langle x\rangle\simeq\Z/6$. Las series de composición
\[
G\supseteq\langle x^2\rangle=G_1\supseteq\{1\},\quad
G\supseteq\langle x^3\rangle=H_1\supseteq\{1\}
\]
son equivalentes por la permutación $\sigma=(12)\in\Sym_2$. Observar que $G/G_1\simeq\Z/3$ y $G/H_1\simeq\Z/2$. 	
\end{example}

Nuestro objetivo es demostrar el teorema de Jordan--Hölder, que afirma que todo grupo que admita una serie de composición, tendrá esencialmente 
una única serie de composición módulo equivalencia. Antes de ir directamente al teorema, necesitamos un resultado previo. 

\begin{lemma}
Sea $G=G_0\supseteq G_1\supseteq \cdots\supseteq G_r=\{1\}$ una serie de composición para $G$ y sea $N$ un subgrupo normal de $G$. Entonces $N$ 
también admite una serie de composición.	
\end{lemma}

\begin{proof}
Para cada $i$, sea $N_i=G_i\cap N$. Como ejercicio, se demuestra que  
\[
N=N_0\supseteq N_1\supseteq\cdots\supseteq N_r=\{1\}
\]
y que además $N_{i+1}$ es normal en $N_i$ para todo $i$. Como
\[
N\cap G_{i+1}=N\cap (G_i\cap G_{i+1})=(N\cap G_i)\cap G_{i+1}
\]
para todo $i$, entonces
\[
\frac{N_i}{N_{i+1}}=\frac{N\cap G_i}{N\cap G_{i+1}}=\frac{N\cap G_i}{(N\cap G_i)\cap G_{i+1}}
\simeq \pi(N\cap G_i)%\unlhd\pi(G_i)=\frac{G_i}{G_{i+1}}.
\]
donde $\pi\colon G_i\to G_i/G_{i+1}$ es el morfismo canónico. En efecto, la restricción $\pi|_{N_i}$ tiene núcleo
$(N_i\cap G_i)\cap G_{i+1}=(N\cap G_{i+1})$ y entonces, por el primer teorema de isomorfismos, $\pi(N_i)=\pi(N\cap G_i)\simeq (N\cap G_i)/(N\cap G_{i+1})$.   
Como $\pi(N\cap G_i)$ es un subgrupo normal del grupo simple $\pi(G_i)=G_i/G_{i+1}$, se sigue que $N_i=N_{i+1}$ o bien $N_i/N_{i+1}=G_i/G_{i+1}$. Luego de remover
las posibles repeticiones obtememos entonces una serie de composición para $N$.   	
\end{proof}

Ahora sí, el teorema.

\begin{theorem}[Jordan--Hölder]
\index{Teorema!de Jordan--Hölder}
Si $G$ es un grupo que admite una serie de composición, entonces todas las series de composición de $G$ tienen la misma longitud y son además equivalentes. 	
\end{theorem}

% Falta la referencia al ejercicio

\begin{proof}
Sean 
\begin{align*}
G=G_0\supseteq G_1\supseteq\cdots\subseteq G_r=\{1\}, &&
G=H_0\supseteq H_1\supseteq\cdots\supseteq H_s=\{1\}	
\end{align*}
dos series de composición de $G$. Procederemos por inducción en $r$. Si $r=1$, entonces $G$ es simple y el teorema queda demostrado trivialmente. Si $r>1$, 
supongamos que el resultado vale para todos los grupos que admiten una series de composición de longitud $<r$. 

Si $G_1=H_1$, entonces $G_1$ admite dos series
de longitudes $r-1$ y $s-1$, respectivamente, y entonces, por hipótesis inductiva, $r=s$ y además las series de composición son equivalentes.  

Si $G_1\ne H_1$, como $G_1$ y $H_1$ son ambos normales en $G$, entonces $G_1H_1$ es también normal en $G$ (esto es un ejercicio que dejamos en la página ??). Como
$G/G_1$ es simple y $G_1\unlhd G_1H_1\unlhd G$, entonces $G_1=G_1H_1$ o bien $G_1H_1=G$, pues $G_1$ es maximal entre todos los subgrupos normales de $G$. Como $G/H_1$ es simple, $H_1$ es maximal entre todos los subgrupos normales de $G$, y entonces $H_1=G_1H_1$ o bien $G_1H_1=G$. En conclusión, $G_1H_1=G$. Sea $K=G_1\cap H_1$.
Entonces $K$ es normal en $G$ y además
\[
G/G_1=\frac{G_1H_1}{G_1}\simeq H_1/K,
\quad
G/H_1=\frac{G_1H_1}{H_1}\simeq G_1/K.
\]
Luego $H_1/K$ y $G_1/K$ son ambos grupos simples. Gracias al lema anterior sabemos que $K$ también admite una serie de composición, digamos
\[
K=K_0\supseteq K_1\supseteq\cdots\supseteq K_t=\{1\}.
\]
Luego 
\[
G_1\supseteq G_2\supseteq\cdots\supseteq G_r=\{1\},
\quad
G_1\supseteq K=K_0\supseteq K_1\supseteq\cdots\supseteq K_t=\{1\},
\]
son ambas series de composición para $G_1$. La hipótesis inductiva implica entonces que $r-1=t+1$, es decir $t=r-2$, y además que estás series de composición son equivalentes. Similarmente, 
\[
H_1\supseteq H_2\supseteq\cdots\supseteq H_s=\{1\},\quad
H_1\supseteq K=K_0\supseteq K_1\supseteq\cdots\supseteq K_t=\{1\}
\]
son series de composición para $H_1$ y además $s-1=t+1=r-1$, lo que implica que $r=s$. Como las series de composición 
\[
G=G_0\supseteq G_1\supseteq K_0\supseteq\cdots\supseteq K_t=\{1\},\quad
G=H_0\supseteq H_1\supseteq K_0\supseteq\cdots\supseteq K_t=\{1\},
\]
son equivalentes, las series
\[
G=G_0\supseteq G_1\supseteq\cdots\supseteq G_r=\{1\},
\quad
G=H_0\supseteq H_1\supseteq\cdots\supseteq H_r=\{1\},
\]
también son equivalentes. 
\end{proof}

Veamos un corolario muy simpático.

\begin{corollary}[teorema fundamental de la aritmética]
	Los primos y sus multiplicidades que aparecen en la factorzación de un entero $n\geq2$ están unívocamente determinados por $n$.
\end{corollary}

\begin{proof}
Escribamos $n=p_1\cdots p_k$, donde los primos $p_1,\dots,p_k$ no son necesariamente distintos. Si $G=\langle x\rangle\simeq\Z/n$ es cíclico de orden $n$, entonces
\[
G=\langle x\rangle\supseteq \langle x^{p_1}\rangle\supseteq \langle x^{p_1p_2}\rangle\supseteq\cdots\supseteq \langle x^{p_1\cdots p_{k-1}}\rangle\supseteq\{1\}
\]
es una serie de composición con factores de orden $p_1p_2,\dots,p_k$, respectivamente. Si $n=q_1\dots q_l$ es otra factorización de $n$ como producto de primos, entonces
\[
G=\langle x\rangle\supseteq \langle x^{q_1}\rangle\supseteq \langle x^{q_1q_2}\rangle\supseteq\cdots\supseteq \langle x^{q_1\cdots q_{l-1}}\rangle\supseteq\{1\}
\]
es también una serie de composición. Por el teorema de Jordan--Hölder, $k=l$ y además las series son equivalentes, algo que se traduce en poder reordenar los primos
$q_1,\dots,q_k$ y obtener como $p_1,\dots,p_k$. 
\end{proof}


 