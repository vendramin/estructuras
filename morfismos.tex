\chapter{Morfismos}

\begin{definition}
	\index{Morfismo!de grupos}
	Sean $G$ y $H$ dos grupos. 
	Una función $f\colon G\to H$ es un \textbf{morfismo de grupos} si 
	$f(xy)=f(x)f(y)$ para todo $x,y\in G$. 
\end{definition}

\index{Morfismo!de grupos inyectivo}
\index{Morfismo!de grupos sobreyectivo}
\index{Morfismo!de grupos biyectivo}
\index{Monomorfismo!de grupos}
\index{Epimorfismo!de grupos}
\index{Isomorfismo!de grupos}
Si un morfismo de grupos es una función inyectiva, se denominará
\textbf{monomorfismo}. Si es una función sobreyectiva, se denominará
\textbf{epimorfismo}.  Si fuera una función biyectiva, \textbf{isomorfismo}.
Dos grupos $G$ y $H$ se dirán \textbf{isomorfos} (la notación será $G\simeq H$)
cuando exista un isomorfismo $G\to H$. 

\begin{examples}\
\begin{enumerate}
\item Si $G$ es un grupo, la función $\id\colon G\to G$ es un morfismo de grupos.
\item Si $G$ y $H$ son grupos, la función $e\colon G\to H$, $e(g)=1_H$, es un morfismo de grupos. 
\item Para cada $n\in\Z$, la función $\Z\to\Z$, $x\mapsto nx$, es un morfismo de grupos.
\item Si $G$ es un grupo abeliano y $n\in\Z$, la función $G\to G$, $g\mapsto g^n$, es un morfismo de grupos.  	
\end{enumerate}
\end{examples}

El siguiente ejemplo es particularmente importante. 

\begin{example}
\index{Morfismo!de conjugación}
\index{Conjugación}
Sea $G$ un grupo y sea $g\in G$. La función $\gamma_g\colon G\to G$, $\gamma_g(x)=gxg^{-1}$, se denomina \textbf{conjugación} por el elemento $g$ y 
es un morfismo de grupos. 	
\end{example}

\begin{example}
La función $\exp\colon\R\to\R^\times$, $\exp(x)=e^x$, es un morfismo de grupos. 
\end{example}

\begin{example}
\index{Inclusión}
La inclusión $\Z\hookrightarrow\Q$ es un morfismo inyectivo de grupos. 
\end{example}

En general, si $S$ es un subgrupo de un grupo $G$, entonces
la \textbf{inclusión} $S\hookrightarrow G$ es un morfismo de grupos. 

\begin{example}
$\det\colon\GL_2(\R)\to\R^\times$ es un morfismo de grupos.
\end{example}

\begin{example}
\index{Restricción de un morfismo}
	Sea $f\colon G\to H$ un morfismo de grupos y sea $S$ un subgrupo de $G$. 
	La \textbf{restricción} $f|_S\colon S\to H$ es también un morfismo de grupos.
\end{example}

\begin{example}
La función $f\colon\R\to\C^\times$, $f(x)=\cos x+i\sin x$, es un morfismo de grupos pues
$f(x+y)=f(x)f(y)$ para todo $x,y\in\R$. 
\end{example}

\begin{exercise}
Sea $f\colon G\to H$ un morfismo de grupos. Demuestre que $f(1)=1$, que $f(g^{-1})=f(g)^{-1}$ y que  $f(g^n)=f(g)^n$ para todo $g\in G$ y $n\in\N$.  
\end{exercise}

\begin{example}
Sea $f\colon\R_{>0}\to\R$, $f(x)=\log(x)$. La fórmula 
\[\log(xy)=\log(x)+
\log(y)
\]
nos dice que $f$ es un morfismo de grupos. Los resultados del ejercicio anterior
se traducen en las siguientes propiedades de la función logaritmo: 
\[
\log(1)=0,
\quad
\log\left(\frac{1}{x}\right)=-\log(x),
\quad
\log(x^n)=n\log(x).
\] 
\end{example}

\begin{definition}
	\index{Núcleo!de un morfismo de grupos}
	Sea $f\colon G\to H$ un morfismo de grupos. El \textbf{núcleo} de $f$ 
	es el conjunto
	$\ker f=\{x\in G:f(x)=1\}$. 
\end{definition}

La propiedad fundamental que tiene el núcleo de un morfismo $f$ es la siguiente: $f(x)=f(y)$ si y sólo si $x=yk$ para algún $k\in\ker f$. 

\begin{example}
Sea $f\colon\mathcal{U}(\Z/21)\to\mathcal{U}(\Z/21)$ el morfismo de grupos definido por $f(x)=x^3$. Entonces 
$\ker f=\{1,4,16\}$ y $f(\mathcal{U}(\Z/21))=\{1,8,13,20\}$. 
\end{example}

%\begin{example}
%Sea $f\colon\Aff(\R)\to\R$ el morfismo $\begin{pmatrix}a&b\\0&1\end{pmatrix}\mapsto a$ que
%vimos en el ejemplo~\ref{exa:afin} de la página~\pageref{exa:afin}. Entonces
%\[
%\ker f=\left\{\begin{pmatrix}
%1&b\\
%0&1
%\end{pmatrix}
%\right\}
%\]
%\end{example}


\begin{example}
\label{exa:afin}
Sea 
\[
\Aff(\R)=\left\{\begin{pmatrix}a&b\\0&1\end{pmatrix}:a\in\R^\times,\,b\in\R\right\}\leq\GL_2(\R).
\]
La función 
\[
f\colon \Aff(\R)\to\R^\times,\quad
\begin{pmatrix}
a&b\\
0&1
\end{pmatrix}
\mapsto a
\]
es un morfismo de grupos (de hecho, $f(x)=\det(x)$ para todo $x\in\Aff(\R)$) tal que
\[
\ker f=\left\{\begin{pmatrix}
1&b\\
0&1
\end{pmatrix}
\right\}.
\]
Dejamos como ejercicio verificar que la función $g\colon\Aff(\R)\to\R$, $\begin{pmatrix}a&b\\0&1\end{pmatrix}\mapsto b$, 
no es un morfismo de grupos. 
\end{example}

\begin{example}
Sea $f\colon\R\to\C^\times$, $f(x)=\cos x+i\sin x$. Entonces
\[
\ker f=\{2\pi k:k\in\Z\}=2\pi\Z.
\] 
\end{example}

\begin{definition}
\index{Imagen!de un morfismo de grupos}
La \textbf{imagen} de $f$ es 
el conjunto $f(G)=\{f(x):x\in G\}$. 
\end{definition}

\begin{proposition}
	Si  $f\colon G\to H$ un morfismo de grupos. Valen las siguientes propiedades:
	\begin{enumerate}
		\item $\ker f$ es un subgrupo normal de $G$.
		\item $f(G)$ es un subgrupo de $H$.
	\end{enumerate}
\end{proposition}

\begin{proof}
	Demostraremos solamente la primera afirmación, la segunda quedará como ejercicio. Primero debemos 
	demostrar que $\ker f$ es un subgrupo de $G$. Para eso, observamos que $1\in \ker f$ y además que si $x,y\in\ker f$ entonces $xy^{-1}\in\ker f$ (pues como $f$ es morfismo de grupos se tiene 
	que $f(xy^{-1})=f(x)f(y)^{-1}=1$). Para verificar que $\ker f$ es normal en $G$, sean $x\in\ker f$ y $g\in G$. Entonces $gxg^{-1}\in\ker f$ pues
	$f(gxg^{-1})=f(g)f(x)f(g)^{-1}=f(g)f(g)^{-1}=1$.   
\end{proof}

La imagen en general no es un subgrupo normal. 

\begin{example}
La inclusión $\langle (12)\rangle\hookrightarrow\Sym_3$ es un morfismo de grupos cuya imagen  
no es un subgrupo normal de $\Sym_3$.
\end{example}

\begin{example}
Sabemos que $\mathcal{U}(\Z/21)=\{1,2,4,5,8,10,11,13,16,17,19,20\}$ es un grupo abeliano. La función $f\colon\mathcal{U}(\Z/21)\to\mathcal{U}(\Z/21)$, $f(x)=x^3$, es un morfismo de grupos. La imagen de $f$ es igual a $\{1,8,13,20\}$, que es un subgrupo de $\mathcal{U}(\Z/21)$. 
\end{example}

\begin{example}
La función $\sgn\colon\Sym_n\to\{-1,1\}$ es un morfismo sobreyectivo de grupos tal que 
$\ker(\sgn)=\Alt_n$. En particular, $\Alt_n$ es un subgrupo normal de $\Sym_n$.   
\end{example}

\begin{example}
\index{Morfismo!canónico}
Si $N$ es un subgrupo normal de $G$, la función $\pi\colon G\to G/N$, $x\mapsto xN$, es un morfismo sobreyectivo tal que $\ker\pi=N$. La función $\pi$ se conoce como 
el \textbf{morfismo canónico} $G\to G/N$.
\end{example}

El ejemplo anterior nos dice, en particular, que cada subgrupo normal de un grupo $G$ es el núcleo de un morfismo con dominio en $G$. 

\begin{exercise}
Sea $f\colon G\to H$ un morfismo de grupos. Demuestre las siguientes afirmaciones:
\begin{enumerate}
\item Si $S\leq G$, entonces $f(S)\leq H$ y además $f^{-1}(f(S))=S\ker f$. 
\item Si $T\leq H$, entonces $\ker f\leq f^{-1}(T)\leq G$ y además $f(f^{-1}(T))=T\cap f(G)$. 
\item $f$ es inyectiva si y sólo si $\ker f=\{1\}$.
\item Si $g\in G$ tiene orden finito, entonces $|f(g)|$ divide a $|g|$. 
\end{enumerate}	
\end{exercise}

%\begin{exercise}
%	Sea $f\colon G\to H$ un morfismo de grupos y sea $Y$ un subconjunto de $H$.
%	Demuestre que $f^{-1}(Y)=\{x\in G:f(x)\in Y\}$ es un subgrupo de $G$. 
%\end{exercise}
%
%Sean $G$ y $H$ grupos. 
%Si existe un isomorfismo de grupos $G\to H$, diremos que $G$ y $H$ son grupos \textbf{isomorfos}. En ese caso, escribiremos
%$G\simeq H$. 
Si $f\colon G\to H$ es un isomorfismo de grupos, entonces $f^{-1}\colon H\to G$ es también un isomorfismo. Observemos además 
que un morfismo de grupos $f\colon G\to H$ será un isomorfismo si y sólo si existe
un morfismo de grupos $g\colon H\to G$ tal que $g\circ f=\id_G$ y $f\circ g=\id_H$. 

\begin{example}
$\Sym_2\simeq\Z/2\simeq G_2$. 	
\end{example}

\begin{example}
$\D_3\simeq\Sym_3$ y el isomorfismo está dado por la función $\D_3\to\Sym_3$, 
\[
1\mapsto \id,\quad 
r\mapsto (123),\quad r^2\mapsto(132),\quad s\mapsto(12),\quad rs\mapsto(13),\quad r^2s\mapsto(23).
\]	
\end{example}

\begin{example}
$\Z/2\times\Z/3\simeq\Z/6$ y el isomorfismo está dado por 
\[
(0,0)\mapsto 0,\quad (1,0)\mapsto 3,\quad
(0,1)\mapsto 4,\quad (1,1)\mapsto 1,\quad (0,2)\mapsto 2,\quad (1,2)\mapsto 5.
\]	
\end{example}

\begin{example}
La función $\log\colon\R_{>0}\to\R$ es un morfismo de grupos. Como $\log$ es biyectiva, 
$\R_{>0}\simeq\R$.
\end{example}

Es fácil demostrar que si $f\colon G\to H$ es un isomorfismo entonces $|g|=|f(g)|$ para todo $g\in G$. 

\begin{example}
$\Z/2\times\Z/2\not\simeq\Z/4$ pues en $\Z/2\times\Z/2$ no hay elementos de orden cuatro. 
\end{example}

\begin{example}
$\Q/\Z\not\simeq\Q$. Ambos son grupos abelianos, pero no son isomorfos. 
Para verlo, primero observamos que en $\Q$ todo elemento no trivial tiene orden 
infinito (pues si $kx=0$ con $k\in\Z$ y $x\in\Q\setminus\{0\}$ entonces $k=0$). 
En cambio, en $\Q/\Z$ todo elemento tiene orden finito. En efecto, si $x=r/s\in\Q$, entonces, como
\[
s(x+\Z)=sx+\Z=r+\Z=\Z
\]
se concluye que $|x+\Z|\leq s$. 
\end{example}

\begin{example}
Veamos que $\mathcal{U}(\Z/5)\simeq\mathcal{U}(\Z/10)$. 
En efecto, ambos grupos son cíclicos de orden cuatro pues 
$\mathcal{U}(\Z/5)=\langle 2\rangle$ y $\mathcal{U}(\Z/10)=\langle 3\rangle$. 
En cambio, $\mathcal{U}(\Z/10)\not\simeq\mathcal{U}(\Z/12)$ pues
en $\mathcal{U}(\Z/12)$ no hay elementos de orden cuatro.  
\end{example}

\begin{exercise}	
Demuestre que $F=\{\sigma\in\Sym_n:\sigma(n)=n\}\leq\Sym_n$ y que $F\simeq\Sym_{n-1}$. 
\end{exercise}

Si $G$ y $H$ son grupos, utilizaremos la siguiente notación:
\[
\Hom(G,H)=\{f\colon G\to H:f\text{ es morfismo}\}.
\]

Veamos algunos ejemplos.

\begin{example}
Veamos que 
$\Hom(\Q,\Z)=\{0\}$. Sea $f\in\Hom(\Q,\Z)$ y sea $p$ un número primo. Si fijamos $x\in\Q$ tenemos entonces, como 
\[
f(x)=f\left(p(x/p)\right)=pf(x/p),
\]	
$p$ divide a $f(x)$, de donde se concluye que $f(x)=0$ para todo $x\in\Q$ pues el primo $p$ es arbitrario.  
\end{example}

\begin{example}
Si $G$ es un grupo, entonces $\Hom(\Z,G)=\{k\mapsto g^k:g\in G\}$. 
Primero observemos que para cada $g\in G$ la función $\Z\to G$, $k\mapsto g^k$, es un morfismo de grupos, pues $k+l\mapsto g^{k+l}=g^kg^l$. Sea $f\in\Hom(\Z,G)$ y sea $g=f(1)$. Si $k>0$, 
\[
f(k)=f(\underbrace{1+\cdots+1}_{k-\text{veces}})=f(1)^k=g^k.
\]
Si, en cambio $k<0$, entonces 
\[
f(k)=f(\underbrace{(-1)+\cdots+(-1)}_{|k|-\text{veces}})=f(-1)^{-k}=(g^{-1})^{-k}=g^k.
\]	
\end{example}

\begin{example}
Vamos a demostrar que $\Hom(\Z/8,\Z/10)$ tiene dos elementos. 
Sea $f\colon\Z/8\to\Z/10$ un morfismo no nulo. Si $n=|f(1)|$, entonces
$n$ divide a $8$, es decir $n\in\{1,2,4,8\}$. Como además $f(1)\in\Z/10$ y $f$ es no nulo, 
$n=2$. Luego $f(1)=5$ y eso define únivocamente al morfismo $f$. En nuestro caso, vemos que  
$f(k)=5k$ para $k\in\{0,1,\dots,7\}$. 	
\end{example}

\begin{exercise}
Calcule $\Hom(\Z/n,G)$ para cualquier grupo $G$. 
%Demuestre que $\Hom(\Z/n,G)=\{k\mapsto g^k:g\in G,\,|g|\text{ divide a }n\}$. 
\end{exercise}

\begin{exercise}
Sean $A$, $B$ y $C$ grupos. Si $f\in\Hom(A,B)$ y $g\in\Hom(B,C)$, 
entonces $g\circ f\in\Hom(A,C)$. 	
\end{exercise}

\begin{exercise}
Demuestre que $\Z/2\times\Z/2$ y $\Z/4$ son los únicos subgrupos de orden cuatro (salvo isomorfismo).
\end{exercise}

Veamos un ejemplo de isomorfismo un poco más difícil que los anteriores.

\begin{example}
Si $G$ es un grupo de orden seis, entonces $G\simeq\Sym_3$ o bien $G$ es cíclico de orden seis. 
Para demostrar nuestra afirmación primero observamos que, como $|G|$ es par, existe en $G$ 
un elemento de orden dos, esto lo vimos en el ejercicio~\ref{xca:orden2}. Si todo elemento 
de $G\setminus\{1\}$ tuviera orden dos, entonces $xy=yx$ para todo $x,y\in G$ y luego
\[
\langle x,y\rangle=\{1,x,y,xy\}\leq G,
\]
una contradicción al teorema de Lagrange. Existe entonces $x\in G$ tal que $x$ tiene orden dos y existe $y\in G\setminus\{1\}$ tal que $y$ no tiene orden dos. Nuevamente el teorema
de Lagrange nos dice que $|y|\in\{3,6\}$ (pues el orden de $y$ es un divisor del orden del grupo $G$). Si $|y|=6$, entonces 
$G\simeq\Z/6$. En cualquier caso, existe $z\in G$ tal que 
$|z|=3$. Tenemos 
\[
\langle x,z\rangle=\{1,x,z,z^2,xz,xz^2\}=G.
\]
Para saber qué grupo es $\langle x,z\rangle$ necesitamos entender el producto $zx$. Sabemos que $zx\in\{xz,xz^2\}$. Si $xz=zx$, entonces $|xz|=6$ (pues $(xz)^k\ne1$ para todo $k\in\{1,\dots,5\}$ y 
además $(xz)^6=1$) y luego 
$G=\langle xz\rangle\simeq\Z/6$. Si, en cambio, estamos en el caso $zx=xz^2$, entonces
$G=\langle x,z:x^2=z^3=1,\,xzx^{-1}=z^2\rangle\simeq\D_3$.      
\end{example}

% \begin{example}
% Si $G$ es un grupo de orden ocho, entonces tenemos cinco posibilidades: $G\simeq(Z/2)^3$, $G\simeq\Z/4\times\Z/2$, $G\simeq \Z/8$, $G\simeq\D_4$ o bien $G\simeq Q_8$. 
% \end{example}

Lo que hicimos hasta ahora nos permite clasificar las clases de isomorfismo de grupos de orden $\leq8$.

\begin{table}[ht]
    \centering
    \begin{tabular}{|c|c|c|}
    \hline
    Orden & Cantidad & Grupos\\
    \hline
        1 & 1 & $\{1\}$ \\
        2 & 1 & $\Z/2$ \\
        3 & 1 & $\Z/3$ \\
        4 & 2 & $\Z/4$, $\Z/2\times\Z/2$ \\
        5 & 1 & $\Z/5$ \\
        6 & 2 & $\Z/6$, $\Sym_3$ \\
        7 & 1 & $\Z/7$ \\
%        8 & 5 & $(\Z/2)^3$, $\Z/4\times\Z/2$, $\Z/8$, $\D_4$, $Q_8$ \\
    \hline
    \end{tabular}
    \caption{Grupos de orden $\leq7$.}
    \label{tab:grupos<8}
\end{table}

\begin{exercise}
\label{xca:size9}
Demuestre que salvo isomorphismo los únicos grupos de orden nueve son $\Z/9$ y $\Z/3\times\Z/3$. 
\end{exercise}

Estamos en condiciones de enunciar y demostrar los teoremas de isomorfismos. 
% Primero comenzaremos con un teorema técnico pero fundamental. 

% \begin{theorem}
% \label{thm:cocientes}
% Sea $f\colon G\to H$ un morfismo de grupos y $K$ un subgrupo normal de $G$ tal que $K\subseteq\ker f$. Existe entonces
% un único morfismo $\varphi\colon G/K\to H$ tal que el diagrama
% \[
%         \xymatrix{
%         G
%         \ar[d]_\pi
%         \ar[r]^f
%         & H
%         \\
%         G/K\ar@{-->}[ur]_{\varphi}
%         }
% \]
% es conmutativo, lo que significa que $\varphi\circ\pi=f$, donde $\pi\colon G\to G/K$ es el morfismo canónico. 
% Más aún, $\ker\varphi=\pi(\ker f)$ 
% y $\varphi(G/K)=f(G)$. En particular, $\varphi$ es inyectiva si y sólo si $\ker f=K$ y $\varphi$ es sobreyectiva si y sólo si $f$ es sobreyectiva. 
% \end{theorem}

% \begin{proof}
% 	Sea $\varphi\colon G/K\to H$, $xK\mapsto f(x)$. Primero debemos demostrar que $\varphi$ está bien definida, lo que significa 
% 	demostrar que si $xK=yK$ entonces $f(x)=f(y)$. En efecto, si $xK=yK$, entonces, como $y^{-1}x\in K$, se tiene que
% 	\[
% 	f(y)^{-1}f(x)=f(y^{-1}x)\subseteq f(K)=\{1\}.
% 	\]
% 	Luego $f(x)=f(y)$. 
		
% 	Veamos que $\varphi$ es morfismo de grupos:
% 	\[
% 	\varphi(xKyK)=\varphi(xyK)=f(xy)=f(x)f(y)=\varphi(xK)\varphi(yK).
% 	\]
% 	Para calcular $\ker\varphi$ procedemos así: 
% 	\[
% 	\pi(x)=xK\in\ker\varphi\Longleftrightarrow \varphi(xK)=1
% 	\Longleftrightarrow f(x)=1
% 	\Longleftrightarrow x\in\ker f.
% 	\]
% 	En consecuencia,  $\ker\varphi=\pi(\ker f)$. 
% 	La igualdad $\varphi(G/K)=f(G)$ es trivial.
	
% 	De la definición de $\varphi$ se obtiene inmediatamente que $\varphi\circ \pi=f$. 
% 	Esta igualdad además garantiza la unicidad del morfismo $\varphi$ pues si $\psi$ 
% 	es tal que $\psi\circ\pi=f$, 
% 	\[
% 	\varphi(xK)=\varphi(\pi(x))=(\varphi\circ\pi)(x)=f(x)
% 	=(\psi\circ\pi)(x)=\psi(\pi(x))=\psi(xK).\qedhere
% 	\]
% \end{proof}

%Como corolario obtenemos:

\begin{theorem}[primer teorema de isomorfismos]
\index{Teorema!de isomorfismos I}
\index{Primer teorema de isomorfismos}	
Si $f\colon G\to H$ es un morfismo de grupos, entonces $G/\ker f\simeq f(G)$. 
\end{theorem}

\begin{proof}	
	Sean $K=\ker f$ y $\varphi\colon G/K\to H$ la función dada por $xK\mapsto f(x)$. Primero debemos demostrar que $\varphi$ está bien definida, lo que significa 
	demostrar que si $xK=yK$ entonces $f(x)=f(y)$. En efecto, si $xK=yK$, entonces, como $y^{-1}x\in K$, se tiene que
	\[
	f(y)^{-1}f(x)=f(y^{-1}x)\in f(K)=\{1\}.
	\]
	Luego $f(x)=f(y)$. 
		
	Veamos que $\varphi$ es morfismo de grupos:
	\[
	\varphi(xKyK)=\varphi(xyK)=f(xy)=f(x)f(y)=\varphi(xK)\varphi(yK).
	\]
	Para calcular $\ker\varphi$ procedemos así: 
	\[
	\pi(x)=xK\in\ker\varphi\Longleftrightarrow \varphi(xK)=1
	\Longleftrightarrow f(x)=1
	\Longleftrightarrow x\in K.
	\]
	En consecuencia,  $\ker\varphi$ es trivial y entonces $\varphi$ es inyectiva. Como es trivial verificar que $\varphi$ es sobreyectica, 
	se concluye que $G/K\simeq f(G)$. 
% 	De la definición de $\varphi$ se obtiene inmediatamente que $\varphi\circ \pi=f$. 
% 	Esta igualdad además garantiza la unicidad del morfismo $\varphi$ pues si $\psi$ 
% 	es tal que $\psi\circ\pi=f$, 
% 	\[
% 	\varphi(xK)=\varphi(\pi(x))=(\varphi\circ\pi)(x)=f(x)
% 	=(\psi\circ\pi)(x)=\psi(\pi(x))=\psi(xK).\qedhere
% 	\]
\end{proof}

\begin{examples}
Si $G$ es un grupo, entonces $G/\{1\}\simeq G$ y $G/G\simeq\{1\}$. 
\end{examples}

\begin{example}
Como $f\colon\Z\to\Z/n$, $x\mapsto x\bmod n$, es un morfismo sobreyectivo con $\ker f=n\Z$, del primer teorema de isomorfismos se concluye que 
$\Z/n\Z\simeq\Z/n$. 	
\end{example}

\begin{example}
Sea $G$ un grupo cíclico infinito, digamos $G=\langle g\rangle$. Es fácil verificar que la función $f\colon\Z\to G$, $k\mapsto g^k$, 
es un isomorfismo de grupos, es decir $G\simeq\Z$. En particular, $G=\langle g^k\rangle$ si y sólo si $k\in\{-1,1\}$. 
\end{example}

\begin{example}
Vamos a demostrar que 
$\Z/n\Z\simeq G_n$. Sea 
\[
f\colon\Z\to G_n,\quad
f(k)=\exp(2i\pi k/n).
\]
Es claro que $f$ es morfismo sobreyectivo y que $\ker f=n\Z$. El resultado que queremos demostrar se obtiene entonces inmediatemente del primer teorema de isomorfismos.	
\end{example}

\begin{example}
Observemos con $2\Z\simeq 3\Z$ (observar que ambos son cíclicos de orden infinito o considerar la función $2k\mapsto 3k$) y que 
\[
\Z/2\simeq\Z/2\Z\not\simeq\Z/3\Z\simeq\Z/3.
\]
\end{example}

\begin{example}
Como 
\[
f\colon\C^\times\to\C^\times, 
\quad
f(z)=\frac{z}{|z|},
\]
es un morfismo tal que $\ker f=\R_{>0}$ y $f(\C^\times)=S^1$, se concluye del primer teorema
de isomorfismos que $\C^\times/\R_{>0}\simeq S^1$.  
\end{example}

\begin{example}
El primer teorema de isomorfismos aplicado a $f\colon S^1\to S^1$, $f(z)=z^2$, permite demostrar que $S^1/\{\pm1\}\simeq S^1$ pues
$\ker f=\{-1,1\}$ y $f(S^1)=S^1$. 	
\end{example}

\begin{example}
Sea $f\colon\C^\times\to\C^\times$, $f(z)=|z|$. Como $\ker f=S^1$ y $f(\C^\times)=\R_{>0}$, se conluye del primer teorema
de isomorfismos que $\C^\times/S^1\simeq\R_{>0}$. 
\end{example}

\begin{example}
Veamos que $(\Z\times\Z)/\langle (1,3)\rangle\simeq\Z$. Para eso, consideramos el morfismo sobreyectivo
$f\colon\Z\times\Z\to\Z$, $f(x,y)=3x-y$. Como 
\[
\ker f=\{(x,3x):x\in\Z\}=\langle (1,3)\rangle,
\]
el primer
teorema de isomorfismos implica que $(\Z\times\Z)/\langle (1,3)\rangle\simeq\Z$. 
\end{example}

\begin{exercise}
Demuestre que $\R/\Z\simeq S^1$.	
\end{exercise}

% $t\mapsto \exp(2\pi it)$
% $z\mapsto z^2$


\begin{exercise}
Demuestre que $\Q/\Z\simeq\cup_{n\geq1}G_n$.
\end{exercise}
% x\mapsto cos (2\pi x)+i\sin (2\pi x)

\begin{exercise}
Demuestre que $(\Z\times\Z)/\langle (6,3)\rangle\simeq\Z\times(\Z/3)$.
\end{exercise}
% (x,y)\mapsto (2y-x,y\mod 3)

\begin{example}
\index{Teorema!fundamental del álgebra lineal}
Si $V$ es un espacio vectorial y $W$ es un subespacio de $V$, entonces, en particular, 
$V$ es un grupo abeliano y $W$ es un subgrupo normal de $V$. El grupo abeliano 
$V/W$ es entonces un espacio vectorial con 
\[
\lambda(v+W)=(\lambda v)+W,\quad \lambda\in\R,\,v\in V,
\]
y el morfismo canónico $\pi\colon V\to V/W$ resulta ser una transformación lineal. 
Dejamos como ejercicio 
demostrar que 
$\dim (V/W)=\dim V-\dim W$
si $\dim V<\infty$. 

Si $f\colon V\to U$ es una transformación lineal, entonces, por el primer teorema de isomorfismos, en particular, 
$V/\ker f\simeq f(V)$ como grupos abelianos. Como el morfismo del primer teorema de isomorfismos es además una transformación lineal, 
se concluye que $V/\ker f\simeq f(V)$ como espacios vectoriales. En particular, si $\dim V<\infty$, entonces 
\[
\dim V-\dim\ker f=\dim f(V).
\]
\end{example}

\begin{exercise}
\label{xca:cocientes}
Sea $f\colon G\to H$ un morfismo de grupos y $K$ un subgrupo normal de $G$ tal que $K\subseteq\ker f$. Demuestre que existe 
un único morfismo $\varphi\colon G/K\to H$ tal que el diagrama
\[\begin{tikzcd}
	G & H \\
	{G/K}
	\arrow["f", from=1-1, to=1-2]
	\arrow["\pi"', from=1-1, to=2-1]
	\arrow["\varphi"', dashed, from=2-1, to=1-2]
\end{tikzcd}
\]
es conmutativo, lo que significa que $\varphi\circ\pi=f$, donde $\pi\colon G\to G/K$ es el morfismo canónico. 
Más aún, $\ker\varphi=\pi(\ker f)$ y $\varphi(G/K)=f(G)$. 
En particular, $\varphi$ es inyectiva si y sólo si $\ker f=K$ y $\varphi$ es sobreyectiva si y sólo si $f$ es sobreyectiva. 
\end{exercise}

El segundo teorema de isomorfismos resultará de gran utilidad al estudiar series de composición y resolubilidad. 
El diagrama que vemos a continuación nos ayudará a recordar cómo funciona el segundo teorema de isomorfismos:
\[\begin{tikzcd}
	& NT \\
	N && T \\
	& {N\cap T}
	\arrow[no head, from=1-2, to=2-3]
	\arrow[no head, from=1-2, to=2-1]
	\arrow[no head, from=2-1, to=3-2]
	\arrow[no head, from=2-3, to=3-2]
\end{tikzcd}\]

\begin{theorem}[segundo teorema de isomorfismos]
\index{Teorema!de isomorfismos II}
\index{Segundo teorema de isomorfismos}	
Si $N$ es un subgrupo normal de $G$ y $T$ es un subgrupo de $G$, entonces $N\cap T$ es normal en $T$ y además 
\[
T/N\cap T\simeq NT/N.
\]	
\end{theorem}

\begin{proof}
Sea $\pi\colon G\to G/N$ el morfismo canónico. Ya vimos que la restricción $\pi|_T\colon T\to G/N$ es un morfismo de grupos con núcleo
$\ker(\pi|_T)=T\cap N$. En particular, $T\cap N$ es normal en $T$. Al aplicar el primer
teorema de isomorfismos, $T/(T\cap N)\simeq \pi(T)$. Como $N$ es normal en $G$, 
$NT$ es un subgrupo de $G$ que contiene a $N$. 
La restricción $\pi|_{NT}$ es entonces un morfismo de grupos con núcleo $NT\cap N=N$.  
Al aplicar el primer teorema de isomorfismos a $\pi|_{NT}$ obtenemos
$NT/N\simeq \pi(NT)=\pi(T)$.  
%%\[
%%T/T\cap N\simeq\pi(T).
%%\]
%%Observemos que $\pi(T)=NT/N$ pues es el subgrupo de $G/N$ formado por las coclases de $N$ en $G$ con representantes en $T$.  
\end{proof}

\begin{exercise}
Sea $N$ normal en $G$ y sea $\pi\colon G\to G/N$ el morfismo canónico. Demuestre que
si $L$ es un subgrupo de $G$, entonces $\pi^{-1}(\pi(L))=NL$. 
%%% 
%%%Por otro lado, si $L$ es un subgrupo de $G$, entonces $\pi^{-1}(\pi(L))=NL$. En efecto,
%%%si $x\in \pi^{-1}(\pi(L))$, entonces $\pi(x)\in \pi(L)$ y luego 
%%%$\pi(x)=\pi(l)$ para algún $l\in L$. Como entonces $xl^{-1}\in \ker\pi=N$, 
%%%se tiene que $x=(xl^{-1})l\in KL$. Recíprocamente, si $x=kl$ con $k\in K$ y $l\in L$, entonces
%%%$\pi(x)=\pi(kl)=\pi(l)\in \pi(L)$ y luego $x\in\pi^{-1}(\pi(L))$. Observemos que
%%%si $L$ es un subgrupo de $G$, entonces $NL$ es un subgrupo de $G$ que contiene a $N$. 
\end{exercise}

En el siguiente ejemplo utilizaremos la notación aditiva. 

\begin{example}
Sea $G=\Z/24$ y sean $H=\langle 4\rangle$ y $N=\langle 6\rangle$. Como $G$ es abeliano, $H$ y $K$ son ambos normales en $G$. Un cálculo directo nos muestra que 
$H+N=\langle 2\rangle$ y que $H\cap N=\{0,12\}$. Notemos que este ejemplo está completamente hecho en la notación aditiva. Calculemos las coclases de $N$ en $H+N$:
\[
0+N=\{0,6,12,18\},
\quad
2+N=\{2,8,14,20\},
\quad
4+N=\{4,10,16,22\}.
\]
Las coclases de $H\cap N$ en $H$ son:
\[
0+(H\cap N)=\{0,12\},
\quad
4+(H\cap N)=\{4,16\},
\quad
8+(H\cap N)=\{8,20\}.
\]
El segundo teorema de isomorfismos nos dice que $(H+N)/N\simeq H/H\cap N$. El isomorfismo 
está dado por $f\colon H/(H\cap N)\to (H+N)/N$, $h+(H\cap N)\mapsto h+N$. 
En nuestro caso, 
\begin{align*}
&f(0+(H\cap N))=0+N,\\
&f(4+(H\cap N))=4+N,\\
&f(8+(H\cap N))=8+N=2+N.
\end{align*}
\end{example}

En los ejemplos que siguen veremos que el segundo teorema de isomorfismos no es algo raro sino que nos permite obtener fórmulas ya conocidas.

\begin{example}
\index{Máximo común divisor}
\index{Mínimo común múltiplo}
Sean $a,b\in\Z$ no nulos. Sabemos que $a\Z+b\Z=\gcd(a,b)\Z$ y que $a\Z\cap b\Z=\lcm(a,b)\Z$. Al aplicar el segundo teorema de isomorfismos, 
\[
\frac{\gcd(a,b)\Z}{b\Z}=\frac{a\Z+b\Z}{b\Z}\simeq
\frac{a\Z}{a\Z\cap b\Z}=\frac{a\Z}{\lcm(a,b)\Z}.
\]
Al aplicar orden, obtenemos la fórmula 
\[
ab=\gcd(a,b)\lcm(a,b).
\] 
\end{example}

\index{Grupo!meta-abeliano}
Veamos otra aplicación. Un grupo $G$ que contiene un subgrupo normal abeliano $N$ y es tal que $G/N$ es abeliano se conoce como grupo \textbf{meta-abeliano}. Claramente, los grupos meta-abelianos no son necesariamente abelianos (el grupo simétrico $\Sym_3$ es meta-abeliano y no abeliano). El segundo teorema de isomorfismos nos permite demostrar que subgrupos de meta-abelianos son meta-abelianos. 

\begin{proposition}
Si $G$ es un grupo meta-abeliano y $H$ es un subgrupo de $G$, entonces $H$ es también meta-abeliano. 
\end{proposition}

\begin{proof}
Como $G$ es meta-abeliano, existe 
un subgrupo normal $N$ de $G$ tal que $N$ y $G/N$ son ambos abelianos. 
El subgrupo abeliano $H\cap N$ es normal en $H$. Gracias al segundo teorema de isomorfismos,
\[
H/(H\cap N)\simeq HN/N
\]
es un grupo abeliano pues $HN/N$ es un subgrupo del grupo abeliano $G/N$.   
\end{proof}

Dejamos la demostración del tercer teorema de isomorfismos como ejercicio. Primero, 
un resultado auxiliar que facilitará los cálculos. 
%Antes de demostrar otro de los teoremas de isomorfismos, necesitamos el siguiente lema técnico. 

\begin{exercise}
	\label{xca:para_3er}
	Sea $f\colon G\to H$ un morfismo de grupos y sean $U\unlhd G$ y $V\unlhd H$. Demuestre que existe 
	un morfismo de grupos $g\colon G/U\to H/V$ tal que el diagrama
\[
\begin{tikzcd}
	G & H \\
	{G/U} & {H/V}
	\arrow["f", from=1-1, to=1-2]
	\arrow["{\pi_U}"', from=1-1, to=2-1]
	\arrow["g"', dashed, from=2-1, to=2-2]
	\arrow["{\pi_V}", from=1-2, to=2-2]
\end{tikzcd}
\]
	es conmutativo si y sólo si $f(U)\subseteq V$, donde $\pi_U\colon G\to G/U$ y $\pi_V\colon H\to H/V$ son los morfismos canónicos. Además, en este caso, 
	\begin{enumerate}
	\item Si $f$ es sobreyectiva, entonces $g$ es sobreyectiva.
	\item Si $U=f^{-1}(V)$, entonces $g$ es inyectiva. 	
	\end{enumerate}
\end{exercise}

% todo: escribir bien la demo del tercero, ese lema es horrible
%Un caso particular del lema nos permite demostrar elegantemente el tercer teorema de isomorfismos.

\begin{exercise}[tercer teorema de isomorfismos]
\label{xca:3er}
\index{Teorema!de isomorfismos III}
\index{Segundo teorema de isomorfismos}	
Sean $S$ y $T$ subgrupos normales de un grupo $G$ tales que $S\subseteq T$. Demuestre que 
entonces $S$ es normal en $T$ 
y $T/S$ es normal en $G/S$. Además
\[
\frac{G/S}{T/S}\simeq G/T,
\]
donde $T/S=\{tS:t\in T\}$.
\end{exercise}

\begin{example}
Si $m$ divide a $n$, entonces $n\Z\leq m\Z\leq\Z$. Luego
\[
\frac{\Z/n\Z}{m\Z/n\Z}\simeq\Z/m\Z.
\]	
\end{example}

El teorema que sigue es también muy importante. Para recordar cómo funciona, podemos hacer uso del siguiente diagrama:
\[
\begin{tikzcd}
	&& G \\
	& L && {f(G)} \\
	N && Y \\
	& {\{1\}}
	\arrow[no head, from=1-3, to=2-4]
	\arrow[no head, from=1-3, to=2-2]
	\arrow[no head, from=2-2, to=3-1]
	\arrow[no head, from=3-1, to=4-2]
	\arrow[no head, from=2-2, to=3-3]
	\arrow[no head, from=3-3, to=4-2]
	\arrow[no head, from=2-4, to=3-3]
\end{tikzcd}
\]

\begin{theorem}[de la correspondencia]
\index{Teorema!de la correspondencia}
Sea $f\colon G\to H$ un morfismo de grupos y sea $K=\ker f$. Existe una correspondencia biyectiva entre
\[\begin{tikzcd}
	{\mathcal{A}=\{L:K\leq L\leq G\}} & {\{Y:Y\leq f(G)\}=\mathcal{B}} 
	\arrow["\sigma", shift left=1, from=1-1, to=1-2]
	\arrow["\tau", shift left=1, from=1-2, to=1-1]
\end{tikzcd}
\]
La correspondencia está dada por $\sigma(L)=f(L)$ y $\tau(Y)=f^{-1}(Y)$. Valen además las siguientes afirmaciones:
\begin{enumerate}
\item $L_1\leq L_2$ si y sólo si $\sigma(L_1)\leq \sigma(L_2)$. 
\item $L\unlhd G$ si y sólo si $\sigma(L)\unlhd f(G)$.
\end{enumerate} 
\end{theorem}

\begin{proof}
	Primero observamos que $\sigma$ y $\tau$ están ambas bien definidas pues vimos en un ejercicio
	que $f(L)\leq f(G)$ y $K\leq f^{-1}(Y)\leq G$. 
	
	Veamos que $\tau\circ\sigma=\id_\mathcal{A}$. Queremos ver que $\tau(\sigma(L))=L$ para todo $L\in\mathcal{A}$. Si $x\in f^{-1}(f(L))$ entonces
	$f(x)\in f(L)$ y luego $f(x)=f(l)$ para algún $l\in L$. Esto implica que $xl^{-1}\in K$ y entonces $x\in Kl\subseteq L$ pues $K\subseteq L$. 
	Recíprocamente, si $l\in L$ entonces $f(l)\in f(L)$ y luego $l\in f^{-1}(f(L))$. 
	
	Veamos que $\sigma\circ\tau=\id_\mathcal{B}$. Si $Y\in\mathcal{B}$, entonces $\sigma(\tau(Y))=Y$. Si $y\in Y\subseteq f(G)$, entonces
	$y=f(x)$ para algún $x\in G$, es decir $x\in f^{-1}(y)$, lo que trivialmente implica que $y=f(x)\in f(f^{-1}(Y))$. Recíprocamente, si $y\in f(f^{-1}(Y))$, entonces
	$y=f(x)$ para $x\in f^{-1}(Y)$. Pero esto significa que $y=f(x)\in Y$.   
	
	Dejamos como ejercicio demostrar que $X\leq Y$ si y sólo si $f(X)\leq f(Y)$. 
	
	Vamos a demostrar que $L\unlhd G$ si y sólo si $f(L)\unlhd f(G)$. Si $L\unlhd G$ y $x\in G$, entonces
	$xLx^{-1}=L$. Esto implica que $f(L)=f(xLx^{-1})=f(x)f(L)f(x)^{-1}$, es decir que $f(L)$ es normal en $f(G)$. Recíprocamente, si 
	$f(L)\unlhd f(G)$ y $x\in G$, entoces 
	\[
	f(xLx^{-1})=f(x)f(L)f(x)^{-1}=f(L).
	\]
	Esto implca que $xLx^{-1}\subseteq LK\subseteq L$ y luego
	$xLx^{-1}\subseteq L$, que implica la normalidad de $L$ en $G$ gracias a la proposición~\ref{pro:normalidad}.
\end{proof}

Veamos una aplicación del teorema anterior. 

\begin{proposition}
	Si $f\colon G\to f(G)$ es un morfismo sobreyectivo de grupos y $H\leq G$ es tal que $K=\ker f\subseteq H$, entonces
	$(G:H)=(f(G):f(H))$. 
\end{proposition}

\begin{proof}
Por el teorema anterior 
sabemos que existe una correspondencia biyectiva
\[
\begin{tikzcd}
	{\{L:K\leq L\leq G\}} & {\{Y:Y\leq f(G)\}}
	\arrow[shift left=1, from=1-1, to=1-2]
	\arrow[shift left=1, from=1-2, to=1-1]
\end{tikzcd}
\]
dada por $H\mapsto f(H)$ e 
inversa dada por $f^{-1}(T)\mapsfrom T$. Sea $H\leq G$ tal que $\ker f\subseteq H$ y sea 
$\alpha\colon G/H\to f(G)/f(H)$ la función dada por $\alpha(gH)=f(g)f(H)$. 
Dejamos como ejercicio verificar que $\alpha$ está bien definida. 
Veamos que $\alpha$ es una función biyectiva pues, en ese caso, 
\[
(G:H)=|G/H|=|f(G)/f(H)|=(f(G):f(H)).
\]

Veamos que $\alpha$ es sobreyectiva: si $yf(H)\in f(G)/f(H)$ entonces
$y=f(g)$ para algún $g\in G$ (pues $f$ es sobreyectiva). Luego 
\[
yf(H)=f(g)f(H)=f(gH)=\alpha(gH).
\]

Veamos ahora que $\alpha$ es inyectiva: si $\alpha(gH)=\alpha(g_1H)$, entonces, por la definición de la función $\alpha$, 
\[
f(g)^{-1}f(g_1)=f(h)\in f(H)
\]
para algún $h\in H$, es decir 
$f(g_1)=f(g)f(h)=f(gh)$ para algún $h\in H$. Esto implica que $g_1=ghk$ para algún $k\in\ker f\subseteq H$ y luego
$g_1=gh_1$ para algún $h_1\in H$, es decir $g_1H=gH$.  
\end{proof}

Es conviente enfatizar qué forma toma el teorema anterior en el caso del morfismo canónico $\pi\colon G\to G/N$.
Si $N$ es un subgrupo normal de $G$, entonces la función $K\mapsto K/N$ es una biyección entre el conjunto de subgrupos (normales) 
de $G$ que contienen a $N$ y el conjunto de subgrupos (normales) de $G/N$. 
Observemos que si $H$ es un subgrupo de $G$, entonces
\[
\pi(H)=HN/N.
\]

\begin{example}
\index{Grupo!de cuaterniones de Hamilton}
Como aplicación del teorema de la correspondencia, vamos a demostrar que todo subgrupo del grupo no abeliano 
\[
Q_8=\{1,-1,i,-i,j,-j,k,-k\}
\]
es normal en $Q_8$. Sea $N=\{-1,1\}$. Entonces $N$ es normal en $Q_8$ (pues $N\subseteq Z(Q_8)$) y además, como $Q_8/N$ tiene cuatro elementos, $Q_8/N$ es un grupo abeliano.

Afirmamos que $N$ está contenido en cualquier subgrupo no trivial de $Q_8$. En efecto, si 
$K$ es un subgrupo no trivial de $Q_8$, entonces $-1\in K$ (pues, por ejemplo, si $-i\in K$, entonces $-1=(-i)^2\in K$). 
Esto implica que cualquier subgrupo de $Q_8$ se corresponde con un subgrupo de $Q_8/N$ y allí todo subgrupo es normal pues $Q_8/N$ es abeliano. Más precisamente, si 
$S\leq Q_8$, entonces $\pi(S)\leq Q_8/N$. Como $Q_8/N$ es abeliano, $\pi(S)$ es normal en $Q_8/N$. 
Como $N\subseteq S$, se tiene que $S=\pi^{-1}(\pi(S))$. Luego $S$ es normal en $Q_8$.
\end{example}

En el ejemplo anterior, podríamos haber demostrado que 
$G/N\simeq\Z/2\times\Z/2$, ya que como sabemos que $|G/N|=4$, hubiera alcanzado con calcular el orden de cada uno de los elementos de $G/N$. 

\begin{example}
Sea $f\colon\Z/12\to\Z/6$ el morfismo dado por $1\mapsto 1$. Un cálculo sencillo nos muestra que $K=\ker f=\{0,6\}$. 
Los subgrupos de $\Z/12$ que contienen a $K$ son 
\[
\langle 1\rangle=\{0,1,\dots,11\},
\quad
\langle 2\rangle=\{0,2,4,6,8,10\},
\quad
\langle 3\rangle=\{0,3,6,9\},
\quad
\langle 6\rangle=\{0,6\},
\] 
que vía $f$ se corresponden con los subgrupos 
\[
\langle 1\rangle=\{0,1,\dots,5\},
\quad
\langle 2\rangle=\{0,2,4\},
\quad
\langle 3\rangle=\{0,3\},
\quad
\{0\}
\]
de $\Z/6$, respectivamente. Por ejemplo, 
\[
\begin{tikzcd}
	&& \Z/12 \\
	& \langle 2\rangle && {\Z/6} \\
	\{0,6\} && \langle 2\rangle \\
	& {\{0\}}
	\arrow[no head, from=1-3, to=2-4]
	\arrow[no head, from=1-3, to=2-2]
	\arrow[no head, from=2-2, to=3-1]
	\arrow[no head, from=3-1, to=4-2]
	\arrow[no head, from=2-2, to=3-3]
	\arrow[no head, from=3-3, to=4-2]
	\arrow[no head, from=2-4, to=3-3]
\end{tikzcd}
\]
\end{example}

Si se tiene un morfismo entre dos grupos,  
en cierto sentido, el teorema de la correspondencia nos permite trasladar propiedades de la
imagen del morfismo al dominio. Veamos una aplicación concreta.

\begin{example}
Sea $G$ un grupo finito que contiene un subgrupo normal $N$ tal que $N\simeq\Z/5$ y $G/N\simeq\Sym_4$. Vamos a demostrar las siguientes afirmaciones sobre $G$.
\begin{enumerate}
\item $|G|=120$
\item $G$ contiene un subgrupo normal de tamaño 20.
\item $G$ contiene tres subgrupos de orden 15, ninguno de ellos normal en $G$.
\end{enumerate}

Para demostrar la primera afirmación usamos el teorema de Lagrange pues
\[
24=|G/N|=\frac{|G|}{|N|}=|G|/5.
\]

Para la segunda afirmación, sea $K$ el subgrupo de $G/N$ isomorfo al grupo de Klein. Entonces
$K$ es normal en $G/N$ y $|K|=4$. Como
$(G/N:K)=6$,  
el subgrupo $K$ de $G/N$ se corresponde con un subgrupo normal $H$ de $G$ de índice 6. El teorema de Lagrange y el teorema de la correspondencia implican entonces que $|H|=20$ pues
\[
6=(G/N:K)=(G:H)=\frac{|G|}{|H|}.
\] 

Para demostrar la tercera afirmación observamos que $G/N\simeq\Sym_4$ tiene cuatro subgrupos de orden 3 (son los subgrupos generados por un 3-ciclo), 
ninguno de ellos normal en $G/N$. Nuevamente, el teorema de la correspondencia, nos dice que estos grupos se corresponderán con 4 subgrupos de $G$, todos de orden 15 y ninguno de ellos normal en $G$.
\end{example}

Recordemos que si $G$ es un grupo, $\Sym_G=\{f\colon G\to G:f\text{ es biyectiva}\}$. 
Terminaremos el capítulo con el siguiente teorema.

\begin{theorem}[Cayley]
Todo grupo $G$ es isomorfo a un subgrupo de $\Sym_G$. 
\end{theorem}

\begin{proof}
Sea $f\colon G\to\Sym_G$, $g\mapsto L_g$, donde $L_g\colon G\to G$, $L_g(x)=gx$. La función $f$ es un morfismo de grupos pues
\[
L_{gh}(x)=(gh)x=g(hx)=L_g(hx)=L_gL_h(x)
\]
para todo $g,h,x\in G$. Además es fácil verificar que $f$ es inyectivo (si $f(g)=f(h)$ entonces $L_g=L_h$, es decir que 
$gx=L_g(x)=L_h(x)=hx$ para todo $x\in G$, que implica que $g=h$). 
\end{proof}

Como aplicación, observamos que todo grupo finito es isomorfo a un subgrupo $\Sym_n$ para algún $n\in\N$. En particular, las matrices de permutación nos permiten observar que
todo grupo finito es un \textbf{grupo lineal}, es decir, isomorfo a un subgrupo de $\GL_n(\Z)$ para algún $n\in\N$. Veamos una aplicación un poquito más sofisticada.
  
\begin{proposition}
Todo grupo simple finito $G$ está contenido en algún $\Alt_n$.
\end{proposition}

\begin{proof}
Si $|G|=2$, el resultado es trivial pues $G\simeq\Alt_2$. Supongamos entonces que $|G|>2$.  
Sea $f\colon G\to\Sym_n$ el morfismo inyectivo obtenido del teorema de Cayley. Si $H=f(G)$, entonces $G\simeq H$ por el primer teorema de isomorfismos. Afirmamos que $H\subseteq\Alt_n$. Si   
$H$ no es un subgrupo de $\Alt_n$, existe $h\in H$ tal que $h\not\in\Alt_n$. Escribimos $h=f(g)$ para algún $g\in G$. Como $h\not\in\Alt_n$, 
\[
\sgn(f(g))=\sgn(h)=-1,
\]
entonces $g\not\in\ker(\sgn\circ f)$. 
Sea $K=\ker(\sgn\circ f)$. Entonces $K=\{1\}$ pues $G$ es simple. Además, $\sgn\circ f$ es una función biyectiva pues $\sgn(f(1))=1$ y $\sgn(f(g))=-1$. En consecuencia,
$G\simeq G/K\simeq\Z/2$, por el primer teorema de isomorfismos.  En particular, $|G|=2$, una contradicción. Luego $H\subseteq\Alt_n$.        
\end{proof}	

Como aplicación simpática del teorema de Cayley puede obtenerse que el axioma de asociatividad en un grupo permite demostrar
que ningún producto necesita llevar paréntesis.  En efecto, el teorema de Cayley afirma que $G$ es un subgrupo de $\Sym_G$. 
La composición de funciones es asociativa y es trivial observar que ninguna composición arbitraria y finita de funciones necesita llevar paréntesis, por eso escribimos 
\[
(f_1\circ\cdots\circ f_n)(g)=f_1(f_2(\cdots f_n(g))\cdots).
\]

