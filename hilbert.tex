\chapter{El teorema de los ceros de Hilbert}

En este capítulo daremos una versión elemental de un
caso particular del teorema de los ceros de Hilbert.
Primero necesitamos un lema.

\begin{lemma}
\begin{enumerate}
    \item Si $\C$ es un subanillo de $R$, entonces $R$ es un espacio vectorial complejo.
    \item Si $M$ es un ideal maximal de $\C[X_1,\dots,X_n]$, entonces $E=\C[X_1,\dots,X_n]/M$ está generado por una cantidad numerable de elementos.
    \item Si $V$ es un espacio vectorial generado por una cantidad numerable de elementos, entonces cada subconjunto de $V$ 
    linealmente independiente es finito o infinito numerable. 
    \item Si $\C(X)=F(\C[X])$ es un espacio vectorial complejo y el conjunto $\{(X-\alpha)^{-1}:\alpha\in\C\}$ es linealmente independiente.
\end{enumerate}
\end{lemma}

\begin{proof}

\end{proof}

Necesitamos además el siguiente lema sobre el cuerpo de fracciones de un dominio. 

\begin{lemma}
Sea $R$ un dominio íntegro. 
\begin{enumerate}
    \item La función $\varphi\colon R\to F(R)$, $x\mapsto \frac{x}{1}$, es un morfismo inyectivo de anillos.
    \item Si $f\colon R\to S$ es un morfismo de anillos tal que $f(x)\in\mathcal{U}(S)$ para todo $x\in R$, entonces existe
    un único morfismo $g\colon F(R)\to S$ de anillos tal que $g\circ\varphi=f$.
    \item Si $\varphi\colon R\to F(R)$ es la inclusión y $E$ es un cuerpo y $f\colon R\to E$ es un morfismo inyectivo de anillos, entonces existe un único morfismo 
    $g\colon F(R)\to E$ de anillos tal que $g|_R=f$. 
\end{enumerate}
\end{lemma}

\begin{proof}

\end{proof}

\begin{theorem}[de los ceros de Hilbert]
\index{Teorema!de los ceros de Hilbert}
Los ideales maximales de $\C[X_1,\dots,X_n]$ están en biyección con 
los puntos de $\C^n$.
\end{theorem}

\begin{proof}
Sea $M$ un idela maximal de $\C[X_1,\dots,X_n]$. Como $E=\C[X_1,\dots,X_n]/M$ es un cuerpo, la restricción
$f_i=\pi|_{\C[X_i]}\colon\C[X_i]\to E$ es un morfismo inyectivo de anillos. Entonces $\ker f_i=\{0\}$ o bien $\ker f_i$ es un ideal maximal. Los ideales 
maximales de $\C[X_i]$ son de la forma $X_i-a_i$ para $a_i\in\C$. 

Afirmamos que $\ker f_i=\{0\}$. De lo contrario...

\end{proof}