\chapter{El lema de Zorn}

\index{Conjunto!parcialmente ordenado}
\index{Poset}
Un conjunto no vacío $R$ se dirá \textbf{parcialmente ordenado} si posee una relación $X$ en $R\times R$ (es decir, $X\subseteq R\times R$)
tal que 
\begin{enumerate}   	
\item $(r,r)\in X$ para todo $r\in R$,
\item $(r,s)\in X$ y $(s,t)\in X$ implican que $(r,t)\in X$, y además
\item $(r,s)\in X$ y $(s,r)\in X$ implican que $r=s$.
\end{enumerate}

La notación que utilizaremos es la siguiente: $(r,s)\in X\Longleftrightarrow r\leq s$. Un conjunto parcialmente ordenado será denotado entonces como el par $(R,\leq)$. Las 
tres condiciones anteriores pueden reescribirse así:
\begin{enumerate}
\item $r\leq r$ para todo $r\in R$.
\item $r\leq s$ y $s\leq t$ implican que $r\leq t$.
\item $r\leq s$ y $s\leq r$ implican que $r=s$.
\end{enumerate}

Otra notación que utilizaremos frecuentemente: $r<s\Longleftrightarrow r\leq s$ y además $r\ne s$.

\index{Elementos!comparables en un poset}
Sea $(R,\leq)$ un conjunto parcialmente ordenado y sean $r,s\in R$. Diremos que $r$ y $s$ son \textbf{comparables}
si $r\leq s$ o bien $s\leq r$. 	

\begin{example}
Sea $U=\{1,2,3,4,5\}$ y sea $T$ el conjunto de subconjuntos de $U$. Definimos 
la relación $C\leq D\Longleftrightarrow C\subseteq D$. Luego $T$ es un conjunto parcialmente ordenado. Los
subconjuntos $\{1,2\}$ y $\{3,4\}$ no son comparables. 
\end{example}

Si $(R,\leq)$ es un conjunto parcialmente ordenado, diremos que un elemento $r\in R$ es \textbf{maximal} en $R$ 
si para todo $t\in R$ comparable con $r$, se tiene que $t\leq r$, es decir: para todo $t\in R$ tal que $r\leq t$ se tiene $r=t$. 	

\begin{example}
$(\Z,\leq)$ no tiene elementos maximales.
\end{example}

\begin{example}
Sea $R=\{(x,y)\in\R^2:y\leq 0\}$. Definimos la relación de orden parcial 
\[
(x_1,y_1)\leq (x_2,y_2)\Longleftrightarrow x_1=x_2\text{ y además }y_1\leq y_2.
\]
Entonces $(R,\leq)$ es un conjunto parcialmente ordenado. Cada elemento de la forma $(x,0)$ 
es un elemento maximal, pues si $(x,0)\leq (x_1,y_1)$, entonces $x=x_1$ y además $y_1=0$. En conclusión, $R$ tiene
una infinidad de elementos maximales. 	
\end{example}


\index{Cota superior}
\index{Lema!de Zorn}
\index{Cadena}
Si $R$ es un conjunto parcialmente ordenado, una \textbf{cota superior} de un subconjunto no vacío $S$ de $R$ 
será un elemento $u\in R$ tal que $s\leq u$ para todo $s\in S$. 


\begin{example}
El conjunto $S=\{6\Z,12\Z,24\Z\}$ de subgrupos de $\Z$ 
es totalmente ordenado con la inclusión. El elemento $6\Z=6\Z\cup 12\Z\cup 24\Z$ es
cota superior de $S$. 	
\end{example}

Un subconjunto no vacío $S$ de $R$ 
será una \textbf{cadena} si dos elementos cualesquiera de $S$ son comparables. El \textbf{lema de Zorn} 
afirma lo siguiente:


\begin{quote}
Si $R$ es un conjunto parcialmente ordenado tal que toda cadena en $R$ admite una cota superior en $R$, entonces
$R$ contiene un elemento maximal.	
\end{quote}

Como se ve, el lema de Zorn nada tiene de intuitivo. Curiosamente, es lógicamente equivalente al axioma de elección y al principio de buena ordenación. 
En realidad, el lema 
de Zorn es un axioma y no un resultado que debe demostrare. 

En vez de profundizar más en los aspectos lógicos del lema de Zorn y sus equivalencias, nos contentaremos con dar una aplicación.

\begin{definition}
\index{Ideal!maximal}
	Sea $R$ un anillo. Diremos que un ideal $I\ne R$ es \textbf{maximal} si dado un ideal $J$ de $R$ tal que 
	$I\subseteq J$, entonces $I=J$ o bien $J=R$.  
\end{definition}

\begin{example}
Si $p$ es un número primo, entonces $p\Z$ es un ideal maximal de $\Z$.
\end{example}

\begin{exercise}
Sea $R$ un anillo. Entonces $R$ es un cuerpo si y sólo si $\{0\}$ es un ideal maximal. 	
\end{exercise}

\begin{exercise}
Sea $R$ un anillo. Un ideal $I$ de $R$ es maximal si y sólo si $R/I$ es un cuerpo.
\end{exercise}

Ahora sí, el teorema.

\begin{theorem}
Sea $R$ un anillo. Todo ideal $I\ne R$ está contenido en un ideal maximal. En particular, todo anillo tiene ideales maximales. 
\end{theorem}

\begin{proof}
Sea 
\[
X=\{J\subseteq R:J\text{ es un ideal tal que }I\subseteq J\subsetneq R\}.
\]
Como $I\in X$, entonces $X$ es no vacío. Luego 
$X$ es un conjunto parcialmente ordenado con la inclusión. Si $C$ es una cadena en $X$, entonces $\cup_{J\in C}J\in X$
es una cota superior para $C$, pues $\cup_{J\in C}J$ es un ideal de $R$ y $1\not\in \cup_{J\in C}J$, lo que
implica que $\cup_{J\in C}J\ne R$. Por el lema de Zorn, existe entonces $M\in X$ un elemento maximal. 

Afirmamos que $M$ es un ideal maximal de $R$. En efecto, si $M_1$ es un ideal propio de $R$ tal que $M\subseteq M_1$, entonces
$I\subseteq M_1$ y luego $M_1\in X$, lo que implica que $M=M_1$ pues $M$ es maximal en $X$.   
\end{proof}

En el teorema anterior, el hecho crucial es la existencia de la unidad del anillo. 
De hecho, el álgebra no conmutativa nos muestra que existen anillos (no son anillos unitarios, obviamente) 
que no poseen ideales maximales.

\begin{example}
Sea $K$ un cuerpo. 
Los ideales maximales de $K[X]$ son los ideales principales generados por los polinomios mónicos irreducibles. En efecto,
si $I$ es un ideal maximal, entonces $I=(f)$ para algún $f\in K[X]$, pues sabemos que $K[X]$ es principal. Si $\deg f=n$ y 
$a$ es el coeficiente principal de $f$, entonces $g=a^{-1}f$ es un polinomio mónico tal que
\[
(a^{-1}f)=(f).
\] 
Podemos suponer entonces, sin perder generalidad, que $f$ es mónico. Si $f$ es irreducible y $(f)\subseteq J\subseteq K[X]$, escribimos
$J=(g)$ para algún $g\in K[X]$. Entonces $f=gh$ para algún $h\in K[X]$, lo que implica que $g$ o $h$ son constantes. Si $g$ es constante, 
entonces $(g)=K[X]$. Si $h$ es constante, digamos $h=a\in K$, entonces $f=ga$, lo que implica que $g=a^{-1}f\in (f)$ y luego $I=J$.   
\end{example}

%\begin{definition}
%\end{definition}
\begin{exercise}
Sea $R$ un dominio de ideales principales. Demuestre que $p\in R$ es irreducible si y sólo si $(p)$ es un ideal maximal. 	
\end{exercise}

Un caso particular del ejercicio anterior. Si $K$ es un cuerpo, $f\in K[X]$ es irreducible si y sólo si el
ideal $(f)$ es maximal.  

\begin{example}
Sean $K$ un cuerpo y $S$ un dominio íntegro. Si $\varphi\in K[X]\to S$ es un morfismo de anillos, entonces $\ker \varphi=\{0\}$ o bien $\ker\varphi$ es maximal. En efecto,
como $K[X]$ es principal, sabemos que $\ker\varphi=(f)$ para algún $f\in K[X]$. Si $\ker\varphi=0$, no hay nada que demostrar. 
Si $f$ es irreducible, entonces $(f)$ es maximal. En caso contrario, 
digamos $f=gh$, tenemos 
\[
0=\varphi(f)=\varphi(gh)=\varphi(g)\varphi(h).
\]
Como $S$ es un dominio íntegro, $\varphi(g)=0$ o bien $\varphi(h)=0$. Sin perder generalidad, podemos suponer que $\varphi(g)=0$,
es decir $g\in\ker \varphi$. Luego $(f)\subseteq (g)=\ker\varphi$ y entonces $h$ es una unidad, una contradicción. 
\end{example}

\begin{example}
El ideal $(X^2+2X+2)$ es maximal en $\Q[X]$ pues 
\[
X^2+2X+2=(X+1)^2+1>0
\]
es irreducible en $\Q[X]$, por ser un polinomio de grado dos sin raíces racionales.  	
\end{example}

\begin{exercise}
Demuestre que $R/I$ es un cuerpo si y sólo si $I$ es un ideal maximal.	
\end{exercise}
	
\begin{example}
Sea $R=(\Z/2)[X]$. Como $X^2+X+1$ es irreducible en $R$, el ideal $I=(X^2+X+1)$ es maximal. Luego $R/I$ es un cuerpo.
\end{example}

\begin{exercise}
\label{xca:Jacobson}
Sea $R$ un anillo conmutativo y sea $J(R)$ la intersección de todos
los ideales maximales de $R$. Pruebe que $x\in J(R)$ si y sólo si
$1-xy\in\mathcal{U}(R)$ para todo $y\in R$. 
\end{exercise}

\begin{exercise}
\label{xca:maxZn}
Los ideales maximales de $\Z/n$ son de la forma $I=Z/p$ donde $p$ es un primo que divide a $n$.
\end{exercise}

Veamos ahora una aplicación a la teoría de grupos. 

\index{Subgrupo!maximal}
Un subgrupo propio $M$ de un grupo $G$ se dice \textbf{maximal} si $M\subseteq H\subseteq G$ para algún subgrupo $H$ de $G$ implica que
$M=H$ o $H=G$, es decir que el subgrupo $M$ es maximal con respecto a la inclusión entre los subgrupos propios de $G$. Quizá sería
mejor definir estos subgrupos como maximal-propio, pero, tal como se hace en la literatura, nos quedaremos con la terminología estándar.  

\begin{exercise}
	Demuestre que $M\leq\Z$ es maximal si y sólo si $M=p\Z$ para algún primo $p$. 
\end{exercise}

\begin{exercise}
    Demuestre que $\Q$ no tiene subgrupos maximales.        
\end{exercise}

\begin{exercise}
        Demuestre que todo subgrupo propio de un grupo finito está contenido en
        algún subgrupo maximal.
\end{exercise}

El resultado del ejercicio anterior puede extenderse a grupos finitamente generados gracias al lema de Zorn. 

\begin{theorem}
Sea $G$ un grupo no trivial y finitamente generado. Todo subgrupo propio de $G$ está contenido en un subgrupo maximal.  	
\end{theorem}

\begin{proof}
	Supongamos que $G=\langle g_1,\dots,g_n\rangle$ y sea $K$ un subgrupo propio de $G$. Para cada $j\in\{0,\dots,n\}$ se define
	$G_j=\langle K,g_1,\dots,g_j\rangle$. Como
	\[
	K=G_0\subseteq G_1\subseteq G_2\subseteq\cdots\subseteq G_n=G,
	\]
	existe entonces $l=\max\{j:0\leq j\leq n-1,\,G_j\ne G\}$. Luego $G_{l+1}=\langle G_l,g_{l+1}\rangle=G$ y
	además $g_{l+1}\not\in G_l$. Sea
	\[
	S=\{H:H\leq G,\,G_l\subseteq H,\,g_{l+1}\not\in H\}
	\]
	ordenado parcialmente con la inclusión. Como $G_l\in S$, entonces $S\ne\emptyset$. Dejamos como ejercicio
	demostrar que si $\{H_i:i\in I\}\subseteq S$ es
	totalmente ordenado, entonces $H=\cup_{i\in I}H_i$ es una cota superior de $S$. Por el lema de Zorn sabemos
	entonces que $S$ tiene un elemento maximal, digamos $M$, es decir que $M$ es maximal con 
	respecto a las siguientes propiedades: $M\leq G$, $G_l\subseteq M$ y $g_{l+1}\not\in M$.   
	
	Vamos a demostrar ahora que $M$ es un subgrupo maximal de $G$ que contiene a $K$. 
	Como $K\subseteq G_l\subseteq M$, entonces $M$ contiene a $K$. Para ver que $M$ es maximal, supongamos que
	$M\leq L\leq G$. Si $g_{l+1}\not\in L$, entonces, por definición, $L	\in S$, pero esto contradice la maximalidad
	del conjunto $M$. Luego $g_{l+1}\in L$, lo que implica que $\langle M,g_{l+1}\rangle\subseteq L$. Como $G_l\subseteq M$, entonces
	\[
	G=G_{l+1}=\langle G_l,g_{l+1}\rangle\subseteq \langle M,g_{l+1}\rangle.   
	\]
	En consecuencia, $G\subseteq L$ y luego $L=G$. 
\end{proof}

