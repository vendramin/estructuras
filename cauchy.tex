\chapter{El teorema de Cauchy}

\index{Puntos fijos!de una acción}
\index{Ecuación de clases}
Si $X$ es un $G$-conjunto finito, sabemos que $X$ puede descomponerse como unión disjunta de órbitas. Sea 
\[
\Fix(X)=\{x\in X:g\cdot x=x\text{ para todo $g\in G$}\}
\]
el \textbf{conjunto de puntos fijos} de $X$. Al tomar cardinal en la descomopsición que tenemos del conjunto $X$, 
agrupar las órbitas que tienen únicamente un elemento y 
utilizar el principio fundamental del conteo para las órbitas con $\geq2$ elementos, obtenemos
\begin{equation}
\label{eq:clases}	
|X|=|\Fix(X)|+\sum_{i=1}^k|G\cdot x_i|
=|\Fix(X)|+\sum_{i=1}^k(G:G_{x_i}),
\end{equation}
donde los $x_j$ son los representantes de las órbitas que tienen $\geq2$ elementos. La fórmula~\eqref{eq:clases} es muy útil y se conoce como
\textbf{ecuación de clases}. 

\begin{example}
Si un grupo finito $G$ actúa en $G$ por conjugación, un cálculo directo nos muestra que $\Fix(G)=Z(G)$ y luego la ecuación de clases queda
\[
|G|=|Z(G)|+\sum_{i=1}^k(G:C_G(x_i)),	
\]
para ciertos $x_1,\dots,x_k\in G$ tales que 
$(G:C_G(x_i))\geq2$ para todo $i\in\{1,\dots,k\}$. 
\end{example}


\begin{definition}
Sea $p$ un número primo. Diremos que $G$ es un $p$-grupo si $|G|=p^m$ para algún $m\in\N_0$.  	
\end{definition}

\begin{theorem}
Sea $p$ un número primo y 
sea $G$ un $p$-grupo. Si $\{1\}\ne N\unlhd G$, entonces $N\cap Z(G)\ne\{1\}$. 	
\end{theorem}

\begin{proof}
Como $N$ es normal en $G$, $G$ actúa en $N$ por conjugación. El teorema fundamental del conteo nos dice que cada órbita de la acción es una potencia del primo $p$. Escribamos
\[
N=\underbrace{\mathcal{O}_1\cup\cdots\cup \mathcal{O}_k}_{\text{órbitas de un elemento}}\cup\underbrace{\mathcal{O}_{k+1}\cup\cdots\cup\mathcal{O}_m}_{\text{órbitas de tamaño $>1$}},
\]	
Como $N\cap Z(G)=\mathcal{O}_1\cup\cdots\cup\mathcal{O}_k$, los números $k=|N\cap Z(G)|$ y $|N\setminus(N\cap Z(G))|$ son divisibles por el primo $p$. Luego
$|N|\equiv|N\cap Z(G)|\bmod p$. Como $1\in N\cap Z(G)$, entonces $|N\cap Z(G)|>1$. En particular, $N\cap Z(G)\ne\{1\}$. 
\end{proof}

\begin{corollary}
Sea $p$ un número primo. Si  
$G$ es un $p$-grupo, entonces $Z(G)\ne\{1\}$.  
\end{corollary}

\begin{proof}
Tomar $N=G$ en el teorema anterior.
\end{proof} 

\begin{corollary}
	Sea $p$ un número primo. Si $|G|=p^2$, entonces $G$ es abeliano.  
\end{corollary}

% todo: página?

\begin{proof}
Por el teorema de lagrange, $|Z(G)|\in\{1,p,p^2\}$. Además, como $G$ es un $p$-grupo, $Z(G)\ne\{1\}$. Si $|Z(G)|=p$, entonces $Z(G)$ es cíclico y luego $G$ es abeliano (es un ejercicio que hicimos en la página...), una contradicción. Luego $|Z(G)|=p^2$ y en consecuencia $G=Z(G)$. 	
\end{proof}

\begin{theorem}[Cauchy]
\index{Teorema!de Cauchy}
Si $G$ es finito y $p$ es un primo que divide al orden de $G$, entonces existe $g\in G$ de orden $p$. 	
\end{theorem}

\begin{proof}
Sea $C=\Z/p$ y sea
\[
X=\{(x_1,\dots,x_p)\in G\times\cdots\times G:x_1\cdots x_p=1\}.
\]
Entonces $C$ actúa en $X$ por $k\cdot (x_1,\dots,x_p)=(x_{k+1},\dots,x_{k+p})$, donde los índices se toman módulo $p$. Para ver que esto es realmente una acción alcanza con observar
que 
\[
x_{i_1}\cdots x_{i_p}=1
\implies (x_{i_1}^{-1}x_{i_1})x_{i_2}\cdots x_{i_p}=x_{i_1}^{-1}
\implies x_{i_2}\cdots x_{i_p}x_{i_1}=1.
\]	
Una vez que $x_1,\dots,x_{p-1}$ están fijos, $x_p=x_{p-1}^{-1}\cdots x_{1}^{-1}$ es la única posibilidad para $x_p$. Luego $|X|=|G|^{p-1}$. Cada $C$-órbita tiene $1$ o $p$ elementos pues $|C|=p$. Escribamos
\[
X=\underbrace{\mathcal{O}_1\cup\cdots\cup \mathcal{O}_k}_{\text{órbitas de un elemento}}\cup\underbrace{\mathcal{O}_{k+1}\cup\cdots\cup\mathcal{O}_m}_{\text{órbitas de tamaño $p$}}.
\] 
Entonces $0\equiv |G|^{p-1}=|X|\equiv k\bmod p$, es decir $p$ divide a $k$. Como además $(1,1,\dots,1)\in X$, $k\geq 1$. Luego $p\leq k$. En particular, 
existe $x\in G\setminus\{1\}$ tal que $(x,x,\dots,x)\in X$. Luego $|x|=p$. 
\end{proof}

\begin{corollary}
	Sea $p$ un primo y $G$ un grupo finito. Entonces $G$ es un $p$-grupo si y sólo si todo elemento de $G$ tiene orden una potencia de $p$. 
\end{corollary}

\begin{proof}
Si $G$ es un $p$-grupo, entonces todo elemento tiene orden una potencia de $p$ por el teorema de Lagrange. Recíprocamente, si $q$ es un primo que divide al orden de $G$, el teorema
de Cauchy nos dice que existe $g\in G$ de orden $q$. Luego $q=p$. 	
\end{proof}

\begin{corollary}
Si $p$ es un primo impar y $G$ es un grupo de orden $2p$, entonces $G\simeq\Z/2p$ o bien $G\simeq\D_p$. 	
\end{corollary}

\begin{proof}
Por el teorema de Cauchy sabemos que existen $r,s\in G$ tales que $|r|=p$ y $|s|=2$. Sea $H=\langle r\rangle$. Entonces $(G:H)=2$ y luego $H\unlhd G$. Escribimos $G=H\cup Hs$ (unión disjunta) pues $s\not\in H$.
En particular,  
\[
G=\{1,r,\dots,r^{p-1},s,rs,\dots,r^{p-1}s\}.
\]
Como $srs^{-1}\in H$, entonces $srs^{-1}=r^k$ para algún $k\in\{0,1,\dots,p-1\}$. Como $s^2=1$,
\[
r=s^2rs^{-2}=s(srs^{-1})s^{-1}=sr^ks^{-1}=r^{k^2}.
\]	
Luego $k^2\equiv 1\bmod p$ y entonces $k\equiv 1\bmod p$ o bien $k\equiv -1\bmod p$. Si $k\equiv -1\bmod p$, entonces $srs^{-1}=r^{-1}$ y luego $G\simeq\D_p$. 
Si $k\equiv 1\bmod p$, entonces $rs=sr$ y luego, como $G$ es abeliano, $G\simeq\Z/{2p}$.
\end{proof}

\begin{theorem}
	Un grupo de orden $p^m$ tiene un subgrupo normal de orden $p^n$ para todo $n\leq m$. 
\end{theorem}

\begin{proof}
Procederemos por inducción en $m$. El caso $m=1$ es trivial. Supongamos entonces que el resultado vale para los grupos de orden $p^m$ y sea $G$ un grupo de orden $p^{m+1}$. 
Queremos ver que si $n\leq m$, entonces $G$ contiene un subgrupo normal de orden $p^n$. Como $Z(G)\ne\{1\}$, existe $g\in Z(G)\setminus\{1\}$ de orden $p$. Sea 
$N=\langle g\rangle\unlhd G$. El grupo $G/N$ tiene orden $p^m$ y entonces, por la hipótesis inductiva, existe un subgrupo normal $Y$ de $G/N$ de orden $p^n$. El teorema de la correspondencia
nos permite afirmar entonces que $G$ contiene un subgrupo normal $K$ de $G$ que contiene a $N$, es decir $N\leq K\leq G$. En efecto, $Y=\pi(K)$ y además 
$(G:K)=(\pi(G):\pi(K))=p^{m-n}$. En consecuencia, $|K|=p^n$.   
\end{proof}


