\chapter{Factorización}

Nuestro objetivo es utilizar ciertos aspectos de la teoría de anillos conmutativos para demostrar algunos resultados de la teoría de números. La idea básica será intentar reconocer similitudes entre
ciertos anillos conmutativos y $\Z$. Comenzaremos entonces 
estudiando divisibilidad en anillos conmutativos. 

\begin{definition}
\index{Dominio!íntegro}
Un anillo conmutativo $R$ será un \textbf{dominio íntegro} si $xy=0\implies x=0$ o bien $y=0$.
\end{definition}

\begin{example}
$\Z$ es un dominio íntegro y $\Z/4$ no lo es.
\end{example}

\begin{example}
\index{Enteros de Gauss}
$\Z[i]$ es un dominio íntegro. Este anillo se conoce como el anillo de \textbf{enteros de Gauss}.
\end{example}

\index{Divisibilidad!en anillos}
\index{Divisor}
\index{Divisor!propio}
Sea $R$ un dominio íntegro.
Vamos a extender algunas nociones de la divisibilidad en $\Z$ 
en el contexto del dominio íntegro $R$. Diremos que
$x$ \textbf{divide} al elemento $y$ si y sólo si $y=xz$ para algún $z\in R$, lo que resulta ser equivalente a pedir $(y)\subseteq (x)$. 
Diremos además que $x$ es un \textbf{divisor propio} de $y$ si y sólo si $(y)\subsetneq (x)\subsetneq R$. Tal como se hace en el caso
de los enteros, a veces utilizaremos la notación $x\mid y$ para referirnos a que el elemento $y$ es divisible por $x$. Escribiremos
$x\nmid y$ cuando $y$ no es divisible por $x$.

\begin{example}
\label{exa:Z[i]div}
Sea $R=\Z[i]$ y sean $d\in\Z$ y $a+bi\in R$. Entonces $d\mid a+bi$ si y sólo si $d\mid a$ y $d\mid b$ en $\Z$. 
En efecto, si existen $e,f\in\Z$ tales que 
\[
a+bi=d(e+fi)=de+dfi,
\] 	
entonces $a=de$ y $b=df$, es decir $d\mid a$ y $d\mid b$. 
\end{example}

Las siguientes definiciones extienden propiedades que conocemos de $\Z$.

\begin{definition}
\index{Elementos!asociados}
Sea $R$ un dominio íntegro y sean $x,y\in R$. Diremos que 	
$x$ e $y$ son \textbf{asociados} si y sólo si $(x)=(y)$.
\end{definition}

Observar que $x$ e $y$ son asociados si y sólo si $x=yu$ para alguna unidad $u$. 

\begin{examples}
En $\Z$ los enteros $2$ y $-2$ son asociados. En $\R[X]$ los polinomios 
$f\ne 0$ y $\lambda f$, donde $\lambda\in\R^\times$, son asociados. 
\end{examples}

\begin{example}
En $\Z[i]$ los elementos $2+5i$ y $-5+2i=(2+5i)i$ son asociados. 
\end{example}

\begin{definition}
\index{Elemento!irreducible}
Sea $R$ un dominio íntegro y sea $x\in R\setminus\{0\}$. Diremos que  
$x$ es \textbf{irreducible} si y sólo si $(x)\ne R$ y $(x)$ es maximal en el conjunto de ideales principales de $R$, es decir que 
no existe ningún ideal principal $(y)$ tal que $(x)\subsetneq (y)\subsetneq R$.
\end{definition}

Para entender mejor la definición de elementos irreducibles observemos que en un dominio íntegro $R$, los divisores de un irreducible 
son sus asociados y las unidades de $R$. En efecto, si $z$ es irreducible y $x\mid z$, entonces $(z)\subseteq (x)$. Esto nos 
da dos posibilidades, $(z)=(x)$ o bien $(x)=R$, es decir $x$ y $z$ son asociados o bien $x\in\mathcal{U}(R)$.   

\begin{example}
Si $K$ es un cuerpo y $f\in K[X]$ de grado $n>0$.
Entonces $f$ es irreducible en $K[X]$ si y solo si los únicos divisores de $f$ son de la forma $g=\lambda$ o bien 
$g=\lambda f$ para $\lambda\in K^\times=K\setminus\{0\}$, es decir cuando los divisores de $f$ 
son unidades de $K[X]$ o los asociados a $f$. 
\end{example}

Del ejemplo anterior se desprende que un polinomio $f$ será reducible (es decir, no irreducible) si $\deg(f)>0$ y además 
$f$ tiene algún divisor $g\in K[X]$ no nulo 
tal que $0<\deg(g)<\deg(f)$.  

\begin{definition}
\index{Elemento!primo}
Sea $R$ un dominio íntegro y sea $p\in R\setminus\{0\}$. Diremos que  
$p$ es \textbf{primo} si y sólo si $(p)\ne R$ y además $xy\in (p)\implies x\in(p)$ o bien $y\in(p)$.   		
\end{definition}

En $\Z$ primos e irreducibles coinciden, algo que no pasará en otros anillos. Más adelante 
caracterizaremos todos los primos en $\Z[i]$. 

\begin{proposition}
En un dominio íntegro, todo elemento primo es irreducible.  
\end{proposition}

\begin{proof}
Como $p$ es primo, $p\ne 0$ y $p\not\in\mathcal{U}(R)$. Sea $x$ un divisor de $p$, digamos $p=xy$ para $y\in R$. Como $xy\in(p)$ y $p$ es primo,
entonces $x\in(p)$ o $y\in(p)$. Si $x\in(p)$, entonces $x=pz$ y luego 
\[
p=xy=(pz)y=p(yz)\implies p(1-yz)=0.
\]
Como $p\ne 0$ y $R$ es un dominio, $y,z\in\mathcal{U}(R)$ y luego $x=pz$ es asociado a $p$. Si $y\in(p)$, una cuenta similar a la anterior nos muestra
que $x\in\mathcal{U}(R)$.
%Sea $x\in R$ un elemento primo. Observemos que $x$ es irreducible si y sólo si $x$ no es una unidad y además no posee divisores propios. Por otro lado, $x$ es primo si y sólo si $x$ no es una unidad y $x\mid yz\implies x\mid y$ o bien $x\mid z$.  
%Si $x$ es primo y además $x=yz$, entonces $x\mid yz$. Esto implica que $x\mid y$ o bien $x\mid z$. Si $x\mid y$, entonces
%$y=xa$ para algún $a$. Luego $x=yz=(xa)z=x(az)$ y entonces $x(1-az)=0$. Como estamos en un dominio íntegro y $x\ne 0$, se concluye
%que $az=1$, es decir $z$ es una unidad.  
\end{proof}

Veremos que la afirmación recíproca no vale con total generalidad, aunque sí en el caso de dominios principales. Para poder hacer algunas
cuentas con mayor facilidad, utilizaremos la siguiente herramienta sobre el anillo $R=\Z[\sqrt{d}]$, donde $d$ es un entero 
libre de cuadrados. Utilizaremos la siguiente notación: si $n\in\N$, entonces $\sqrt{-n}=\sqrt{n}i$. 

\begin{lemma}
Sea $d\in\Z$ libre de cuadrados y sea $N(a+b\sqrt{d})=|a^2-db^2|$. 
\begin{enumerate}
\item $N(\alpha)=0\Longleftrightarrow \alpha=0$.
\item $N(\alpha\beta)=N(\alpha)N(\beta)$ para todo $\alpha,\beta\in\Z[\sqrt{d}]$. 
\item $\alpha\in\Z[\sqrt{d}]$ es una unidad $\Longleftrightarrow N(\alpha)=1$.
\item Si $N(\alpha)$ es primo, entonces $\alpha$ es irreducible en $\Z[\sqrt{d}]$.  	
\end{enumerate}
\end{lemma}

\begin{proof}
Dejamos las primeras dos afirmaciones como ejercicio. Demostremos (3). Si $\alpha\in\mathcal{U}(\Z[\sqrt{d}])$, 
entonces $\alpha\beta=1$ para algún $\beta\in\Z[\sqrt{d}]$. Como
\[
1=N(1)=N(\alpha\beta)=N(\alpha)N(\beta),
\]
entonces $N(\alpha)=1$. Recíprocamente, si $\alpha\ne0$, entonces $\alpha$ es una unidad con inversa
$\alpha^{-1}=\overline{\alpha}/N(\alpha)$, 
donde $\overline{a+b\sqrt{d}}=a-b\sqrt{d}$.

Demostremos ahora (4). Si $\alpha=\beta\gamma$, entonces $N(\alpha)=N(\beta\gamma)=N(\beta)N(\gamma)$ y luego
$N(\beta)=1$ o bien $N(\gamma)=1$ (pues $N(\alpha)$ es primo). Luego
$\beta\in\mathcal{U}(\Z[\sqrt{d}])$ o bien $\gamma\in\mathcal{U}(\Z[\sqrt{d}])$ por el ítem anterior.
\end{proof}

\begin{example}
$\mathcal{U}(\Z[i])=\{-1,1,i,-i\}$ pues si $a+bi\in\Z[i]$ es una unidad, entonces $N(a+bi)=a^2+b^2=1$. 	
\end{example}

\begin{example}
Veamos que $1+\sqrt{-5}$ es irreducible en $\Z[\sqrt{-5}]$. Si $1+\sqrt{-5}=\alpha\beta$, entonces 
$6=N(1+\sqrt{-5})=N(\alpha)N(\beta)$. Esto implica que $N(\alpha)\in\{1,2,3,6\}$. Si $N(\alpha)\in\{1,6\}$, entonces
$\alpha$ es una unidad o $\beta$ es una unidad. Supongamos entonces 
$\alpha=x+y\sqrt{-5}$. Entonces 
$x^2+5y^2=N(\alpha)\in\{2,3\}$, una contradicción. De la misma forma puede demostrarse que $1-\sqrt{-5}$ 
es irreducible. 

Veamos que los números 
$2$, $3$, $1+\sqrt{-5}$ y $1-\sqrt{-5}$ no son asociados. En efecto, $2$, $3$ y $1+\sqrt{-5}$ no son asociados
pues tienen todos distinta norma. Tampoco son asociados $1+\sqrt{-5}$ y $1-\sqrt{-5}$, pues la igualdad  
$1+\sqrt{-5}=u(1-\sqrt{-5})$ implica que $u\in\mathcal{U}(\Z[\sqrt{-5}])=\{-1,1\}$. 
\end{example}

\begin{example}
Sea $R=\Z[i]$. 
Veamos que $3$ es irreducible en $R$. Si $3=\alpha\beta$, entonces
$9=N(\alpha\beta)=N(\alpha)N(\beta)$. Luego $N(\alpha)\in\{1,3,9\}$. Supongamos que $\alpha=a+bi$. Si $N(\alpha)=3$, entonces $|a^2+b^2|=3$, una contradicción pues $a,b\in\Z$. Luego $N(\alpha)\in\{1,9\}$. Si $N(\alpha)=1$, entonces $\alpha$ es una unidad. Si no, $N(\beta)=1$ y $\beta$ es una unidad. 	

Veamos ahora que $2\in\Z[i]$ no es irreducible. En efecto, alcanza con observar que $2=(1+i)(1-i)$, que 
$1\pm i$ no son unidades pues $N(1\pm i)=2\ne 1$. 
\end{example}

\begin{example}
Sea $R=\Z[\sqrt{-3}]$ y sea $x=1+\sqrt{-3}$. Entonces $x$ es irreducible. En efecto,
si $1+\sqrt{-3}=\alpha\beta$, entonces $4=N(\alpha)N(\beta)$. Si $x=a+b\sqrt{-3}$, entonces
$a^2+3b^2=2$. Obviamente, los enteros $a$ y $b$ tienen que tener la misma paridad. 
Si $a\equiv b\equiv 0\bmod 2$, digamos $a=2k$ y $b=2l$, entonces
\[
2=a^2+3b^2=(2k)^2+3(2l)^2=4k^2+12l^2
\]
es divisible por $4$, una contradicción. Si $a\equiv 1\bmod 2$ y además $b\equiv 1\bmod 2$, digamos
$a=2k+1$ y $b=2l+1$, entonces
\[
2=a^2+3b^2=(2k+1)^2+3(2l+1)^2=4k^2+4k+12l^2+12l+4
\]
es un múltilo de $4$, una contradicción. 

Sin embargo, $x$ no es primo. Como
\[
(1+\sqrt{-3})(1-\sqrt{-3})=4,
\]
$x$ divide a $4=2\times 2$, pero $1+\sqrt{-3}$ no divide a $2$ pues 
\[
(1+\sqrt{-3})(a+b\sqrt{-3})=2\implies
(a-3b)+(a+b)\sqrt{-3}=2
\]
que implica que $a-3b=2$ y además $a+b=0$. En conclusión, $a=1/2\not\in\Z$, una contradicción. 
\end{example}

\begin{exercise}
Demuestre que $1+\sqrt{5}\in\Z[\sqrt{5}]$ es un irreducible que no es primo.	
\end{exercise}

\begin{definition}
\index{Dominio!principal}
\index{Dominio!de ideales principales}
Un dominio íntegro $R$ se dirá un \textbf{dominio de ideales principales} (o simplemente dominio principal) si 
todo ideal de $R$ es principal. 	
\end{definition}

Vimos que $\Z$ es un dominio de ideales principales. Si $K$ es un cuerpo, entonces
$K[X]$ es también un dominio de ideales principales.

\begin{example}
Veamos que $\Z[X]$ no es principal. Sea $I=(X,2)$.
Primero observemos que $I\ne\Z[X]$. En efecto, si $I=\Z[X]$, entonces
\[
1=2f+Xg
\]
para ciertos $f,g\in\Z[X]$. En particular, al utilizar el grado, $0=\deg f+1+\deg g$, una contradicción. 
Si existe $h\in\Z[X]$ tal que $I=(h)$, entonces, en particular, $2=hg$ y además $X=hf$ para ciertos $f,h\in\Z[X]$. 
En particular, $\deg h=0$ y luego, como $2=h(1)g(1)$, vemos que $h(1)\in\{-2,2\}$ pues $h(1)\not\in\{-1,1\}$ ya que $\pm1\not\in I$. 
Esto implica que $X=\pm 2f$. Si escribimos 
$f=a_0+a_1X+\cdots+a_nX^n$, donde $a_0,a_1,\dots,a_n\in\Z$, entonces, al comparar el coeficiente de $X$ vemos que 
$1=\pm 2a_1$, una contradicción pues $a_1\in\Z$. 
\end{example}

\begin{example}
Veamos que $\Z[\sqrt{-5}]$ no es principal. Sea $I=(2,1+\sqrt{-5})$. Primero
observamos que $I\ne\Z[\sqrt{-5}]$. En efecto, si $I=\Z[\sqrt{-5}]$, entonces
\[
1=2(x+\sqrt{-5}y)+(1+\sqrt{-5})(u+\sqrt{-5}v)=(2x+u-5v)+\sqrt{-5}(2y+u+v)
\]
para ciertos $x,y,u,v\in\Z$. Luego
\[
1=2x+u-5v,\quad
0=2y+u+v,
\]
que implica que $1=2(x+y+u-2v)$, una contradicción pues $x+y+u-2v\in\Z$. 
Si $I=(x)$, entonces $x\mid 2$ y además $x\mid 1+\sqrt{-5}$. Luego
$N(x)\mid N(2)=4$ y además $N(x)\mid N(1+\sqrt{-5})=6$. Si $x=a+b\sqrt{-5}$, entonces $N(x)=a^2+5b^2$. 
Luego $x$ es una unidad, pues $N(x)=1$, y entonces $I=\Z[\sqrt{-5}]$.  
\end{example}

\begin{proposition}
Sea $R$ un dominio principal y sea $x\in R$. Entonces $x$ es irreducible si y sólo si $x$ es primo.
\end{proposition}

\begin{proof}
	Vimos en la proposición anterior que todo primo es irreducible. Supongamos entonces que $x$ es irreducible y
	que $x\mid yz$. Sea $I=(x,y)$ el ideal generado por $x$ e $y$. Como $R$ es principal, existe $a\in R$ tal que
	$I=(a)$. En particular, $x=ab$ para algún $b\in R$. La irreducibilida de $x$ implica que $a\in\mathcal{U}(R)$ o bien
	$b\in\mathcal{U}(R)$. Si $a\in\mathcal{U}(R)$, entonces $I=R$ y luego $1=xr+ys$ para ciertos $r,s\in R$, lo que 
	implica que
	\[
	z=z1=z(xr+ys)=xzr+yzs
	\]
	y entoces $x\mid z$. Si $b\in\mathcal{U}(R)$, entonces $I=(x)=(a)$ y luego, como $y\in I$, existe
	$t\in R$ tal que $xt=y$, es decir $x\mid y$. 	
\end{proof}
	
\begin{example}
$\Z[\sqrt{-3}]$ 
no es principal ya que existen irreducibles que no son primos.
\end{example}

\begin{example}
Como $\Z$ es principal, en $x\in \Z$ es primo si y sólo si $x\in\Z$ es irreducible.  
\end{example}

\begin{definition}
\index{Dominio!euclidiano}
Sea $R$ un dominio íntegro. Diremos que $R$ es un \textbf{dominio euclidiano} 
si existe una función $\varphi\colon R\setminus\{0\}\to\N_0$ 
tal que para cada $x,y\in R$ con $y\ne0$ existen $q,r\in R$ tales que $x=yq+r$, donde $r=0$ o bien $\varphi(r)<\varphi(y)$.  
\end{definition}

Es importante remarcar que en la definición de dominio euclidiano no pedimos la unicidad que
tenemos en $\Z$. En muchos libros de texto, en la definición de dominio euclidiano, a la función $\varphi$ se le pide además 
que cumpla $\varphi(x)\leq\varphi(xy)$ para todo $x,y\in R\setminus\{0\}$. Esta condición no se usa para demostrar los resultados básicos, 
por eso no se incluye. Además,
si $R$ es euclidiano (con nuestra definición y nuestro $\varphi$), siempre puede reemplazarse $\varphi$ por una función $\psi(x)=\min_{y\ne0}\varphi(xy)$ y esta $\psi$ satisface 
$\psi(x)\leq\psi(xy)$ para todo $x,y\in R\setminus\{0\}$. 


\begin{examples}\
\begin{enumerate}
\item $\Z$ es un dominio euclidiano con $\varphi(x)=|x|$. Cuidado que acá tampoco tenemos unicidad. 
\item Si $K$ es un cuerpo, $K[X]$ es euclidiano con $\varphi(f)=\deg f$. Este ejemplo es el que motiva que la función $\varphi$ de la definición de dominio euclidiano 
esté definida para elementos no nulos del anillo. 
\end{enumerate}
\end{examples}

Veamos otro ejemplo de dominio euclidiano.	
	
\begin{example}
Sea $\Z[i]=\{a+bi:a,b\in\Z\}$. Dejamos como ejercicio demostrar que $\Z[i]$ es un dominio íntegro. 

Sea $N(x+iy)=x^2+y^2$. Veamos que $N$ es un función multiplicativa: Si $\alpha=x+iy$ y $\beta=u+iv$, entonces 
\begin{gather*}
\alpha\beta=(xu-yv)+i(xv+yu)
\shortintertext{y luego}
N(\alpha\beta)=(xu-yv)^2-(xv+yu)^2=(x^2+y^2)(u^2+v^2)=N(\alpha)N(\beta).
\end{gather*}
	
Vamos a demostrar ahora que $\Z[i]$ es un dominio euclidiano con $\varphi(\alpha)=N(\alpha)$. 
Sean $\alpha=a+ib$ y $\beta=c+id\ne 0$. Entonces
\[
\frac{\alpha}{\beta}=\frac{a+bi}{c+di}=r+is,
\]
donde $r=(ac+bd)/(c^2+d^2)$ y $s=(bc-ad)/(c^2+d^2)$. 
Sean $m,n\in\Z$ tales que $|r-m|\leq 1/2$ y $|s-n|\leq 1/2$. Si $\delta=m+in$ y $\gamma=\alpha-\beta\delta$, entonces
$\delta,\gamma\in\Z[i]$ y además $\alpha=\beta\delta+\gamma$. Si $\gamma\ne0$, entonces
\begin{align*}
\varphi(\gamma)&=\varphi\left(\beta\left(\frac{\alpha}{\beta}-\delta\right)\right)
=\varphi(\beta)\varphi\left(\frac{\alpha}{\beta}-\delta\right)\\
&=\varphi(\beta)\varphi( (r-m)+i(s-n))
=\varphi(\beta)\left( (r-m)^2+(s-n)^2\right)\\
&\leq\varphi(\beta)\left(\frac14+\frac14\right)=\frac12\varphi(\beta)<\varphi(\beta).
\end{align*}
\end{example}

En $\Z[i]$ hay algoritmo de división pero no tenemos unicidad. De hecho, 
por ejemplo, podemos escribir 
\[
1+8i=(2-4i)(-1+i)+(-1+2i)
=(2-4i)(-2+i)+(1-2i)
\] 
y entonces
$N(1-2i)=N(-1+2i)=5<20=N(2-4i)$.

\begin{example}
Si $\omega=\frac{-1+\sqrt{3}i}{2}$, entonces $\Z[\omega]$ es un dominio euclidiano.	Primero observamos que
\[
\Z[\omega]=\left\{\frac{a}{2}+\frac{b}{2}\sqrt{-3}:a,b\in\Z,\,a\equiv b\bmod 2\right\}
\]
pues $a+b\omega=\frac{2a+b}{2}+\frac{b}{2}\sqrt{-3}$. 

Veamos que $\Z[\omega]$ es euclidiano con la norma  
$N(a+b\sqrt{-3})=a^2+3b^2$, donde $a,b\in\frac{1}{2}\Z$. Sean $\alpha=a_1+a_2\omega$ y $\beta=b_1+b_2\omega\ne 0$, donde
$a_1,a_2,b_1,b_2\in\frac{1}{2}\Z$. Queremos ver que existen $\gamma,\delta\in\Z[\omega]$ tales que $\alpha=\beta\gamma+\delta$, donde $N(\delta)<N(\beta)$. 
En $\Q[\sqrt{-3}]$ podemos dividir, entonces 
\[
\frac{\alpha}{\beta}=c_1+c_2\sqrt{-3},
\]
para ciertos $c_1,c_2\in\Q$. Sea $q_2\in\Z$ tal que $|2c_2-q_2|\leq 1/2$ y sea $t\in\Z$ el entero más cercano al número $c_1-\frac{q_2}{2}$. 
Si $q_1=2t+q_2$, entonces $|c_1-\frac{q_1}{2}|\leq 1/2$. Si 
\[
\gamma=\frac{q_1}{2}+\frac{q_2}{2}\sqrt{-3},
\]
entonces $\gamma\in\Z[\omega]$ pues $q_1-q_2=2t$ es par. Si 
\[
\delta=\beta\left( (c_1-\frac{q_1}{2})+(c_2-\frac{q_2}{2})\sqrt{-3}\right),
\]
entonces $\alpha=\gamma\beta+\delta$. 
Como  
\begin{multline*}
N\left( (c_1-\frac{q_1}{2})+(c_2-\frac{q_2}{2})\sqrt{-3}\right)
\leq ( c_1-\frac{q_1}{2})^2+3(c_2-\frac{q_2}{2})^2\leq \frac{1}{4}+3\frac{1}{16}<1, 
\end{multline*}
se concluye que $N(\delta)<N(\beta)$ pues $N$ es multiplicativa. 
\end{example}

\begin{exercise}
Demuestre que $\Z[\sqrt{d}]$ es euclidiano si $|d|\leq 2$.
\end{exercise}

El siguiente ejemplo no es sencillo. Simplemente lo mencionamos para mayor completitud en la presentación.  Para más información ver \cite{MR967349,MR3665445,MR314831}.

\begin{example}
Si $\theta=\frac{1+\sqrt{-19}}{2}$, entonces $\Z[\theta]$ no es euclidiano.	Sin embargo, $\Z[\theta]$ es 
un dominio de ideales principales. 
\end{example}

Tal como pasa en $\Z$ y $K[X]$, el tener algoritmo de división nos permite demostrar que todo ideal es principal.

\begin{theorem}
	Si $R$ es euclidiano, entonces $R$ es principal. 
\end{theorem}

\begin{proof}
	Sea $I$ un ideal no nulo de $R$ y sea $y\in I$ donde la función $\varphi(x)$ con $x\in I\setminus\{0\}$ alcanza su mínimo. 
	Si $z\in I$, entonces $z=yq+r$ donde $r=0$ o bien $\varphi(r)<\varphi(y)$. La minimalidad de $y$ implica que $r=0$ y 
	luego $z=yq$. Tenemos entonces $I\subseteq Ry\subseteq (y)\subseteq I$ y luego $I=(y)$.     
\end{proof}

Nos interesa poder reconocer anillos de la forma $\Z[\sqrt{d}]$, con $d$ libre de cuadrados, que se parezcan al anillo $\Z$. Ya vimos
que hay muchas similitudes, pero en $\Z[\sqrt{d}]$ no siempre valdrá el teorema de la factorización única.

\begin{definition}
\index{Dominio!de factorización única}
Diremos que un dominio íntegro $R$ es un \textbf{dominio factorización única} si valen las siguientes propiedades:
\begin{enumerate}
\item Cada $x\ne0$ que no es una unidad puede escribirse 
como $x=c_1\dots c_n$ para ciertos irreducibles $c_1,\dots,c_n$.  
\item Si $x=c_1\cdots c_n=d_1\cdots d_m$ con los $c_i$ y los $d_j$ irreducibles, entonces $n=m$ y además existe 
una permutación $\sigma\in\Sym_n$ tal que $c_i$ y $d_{\sigma(i)}$ son asociados para todo $i\in\{1,\dots,n\}$. 	
\end{enumerate}
\end{definition}

El ejemplo típico de domino de factorización única es $\Z$. 

Intentaremos explicar mejor la diferencia entre tener factorización y tener factorización única. Supongamos que $R$ es un dominio íntegro noetheriano. Podemos demostrar entonces que $R$ tendrá factorización, aunque no necesariamente única. En efecto, si $x\in R$ no admite una factorización, entonces $x$ no será 
producto de irreducibles. Podremos escribir entonces $x=yz$, donde $y$ o bien $z$ no es factorizable. Al repetir este procedimiento 
nos encontramos con una sucesión de la forma
\[
(x)\subsetneq (x_1)\subsetneq (x_2)\subsetneq\cdots
\]
que no se estabiliza, una contradicción a la noetherianidad. 

\begin{example}
Gracias al teorema de Hilbert sabemos que $\Z[i]$ y $\Z[\sqrt{-6}]$ son ambos noetherianos, lo que nos dice que 
en en esos anillos existirá factorización. Sin embargo, veremos que en $\Z[i]$ hay factorización única y que en $\Z[\sqrt{-6}]$ no. 
\end{example}

\begin{theorem}
Sea $R$ un dominio de ideales principales. 
Entonces $R$ es un dominio de factorización única.
\end{theorem}

\begin{proof}
Primero demostraremos que $R$ es noetheriano. Como todos los ideales de $R$ son principales, toda sucesión de ideales es
de la forma $(a_1)\subsetneq (a_2)\subsetneq\cdots$. Fijada esa sucesión, la unión 
$J=\cup_{i\geq 1}(a_i)$ es un ideal de $R$. Como $R$ es principal, $J=(x)$ para algún $x\in R$. En particular, como $x\in (a_i)$ para algún $i\geq1$, 
podemos concluir que $(x)\subseteq (a_i)$. Por otro lado, $(a_i)\subseteq (a_{i+1})\subset (x)$. Luego
$(x)=(a_i)=(a_{i+1})$ y entonces la sucesión de ideales principales se estabiliza. 

Sabemos que en $R$ todo irreducible es primo. Sea $x\in R$. Gracias a la noetherianidad de $R$ sabemos que $x$ puede factorizarse. Supongamos que 
\[
x=c_1\cdots c_n=d_1\cdots d_m
\]
son factorizaciones de $x$ en irreducibles, donde $n\leq m$. 
Si $m=1$, entonces $n=1$ y luego $c_1=d_1$. Si $m>1$, como $c_1$ es primo (pues sabemos 
que en $R$ los irreducibles son primos) y además $c_1\mid d_1\cdots d_m$, 
entonces $c_1$ divide a alguno de los $d_j$, digamos $c_1\mid d_1$, sin perder generalidad. 
Como $d_1$ es irreducible y $c_1\not\in\mathcal{U}(R)$, 
$c_1$ y $d_1$ son asociados, es decir $c_1=ud_1$ para algún $u\in\mathcal{U}(R)$. Como entonces 
\[
c_1c_2\cdots c_n=(ud_1)c_2\cdots c_m=d_1d_2d_3\cdots d_m,
\]
se sigue que 
\[
d_1(c_2\cdots c_n-u^{-1}d_2\cdots d_m)=0.
\]
Como $R$ es un dominio y además $d_1\ne 0$, 
después de reemplazar, sin perder generalidad, $u^{-1}d_2$ por $d_2$, nos quedamos con $c_2\cdots c_n=d_2\cdots d_m$. Por inducción, queda entonces
demostrada la implicación que queríamos probar.
% Supongamos ahora que $R$ es un dominio de factorización única. Si $p$ es irreducible y no es primo, existen $x,y\in R$ tales que
% $p\mid xy$, $p\nmid x$ y $p\nmid y$. Como $xy=pz$ para algún $z\in R$, al factorizar $x$, $y$ y $z$ en irreducibles, obtenemos dos factorizaciones
% distintas para $xy$, una contradicción a la unicidad de la factorización. 	
\end{proof}

% \begin{corollary}
% 	Un dominio de ideales principales es un dominio de factorización única.
% \end{corollary}

% \begin{proof}
% Sea $R$ un dominio de ideales principales. Vimos que en $R$ primos e irreducibles son equivalentes. Por el teorema anterior 
% basta con demostrar que $R$ es noetheriano. Como todos los ideales de $R$ son principales, toda sucesión de ideales es
% de la forma $(a_1)\subsetneq (a_2)\subsetneq\cdots$. Fijada esa sucesión, la unión 
% $J=\cup_{i\geq 1}(a_i)$ es un ideal de $R$. Como $R$ es principal, $J=(x)$ para algún $x\in R$. En particular, como $x\in (a_i)$ para algún $i\geq1$, 
% podemos concluir que $(x)\subseteq (a_i)$. Por otro lado, $(a_i)\subseteq (a_{i+1})\subset (x)$. Luego
% $(x)=(a_i)=(a_{i+1})$ y entonces la sucesión de ideales principales se estabiliza. 
% \end{proof}

\begin{example}
$\Z[\sqrt{-6}]$ no es un dominio de factorización única. En efecto, primero observamos que 
\[
10=2\cdot 5=(2+\sqrt{-6})(2-\sqrt{-6}).
\]
Recordemos que $N(a+b\sqrt{-6})=a^2+6b^2$.	
Primero veamos que $2$ es irreducible. Si $2=\alpha\beta$ con $\alpha,\beta\not\in\mathcal{U}(R)$, entonces 
\[
4=N(2)=N(\alpha)N(\beta)
\]
y luego $N(\alpha)=N(\beta)=2$, una contradicción pues $a^2+6b^2\ne 2$ para todo $a,b\in\Z$. 
De la misma forma se demuestra que $5$ 
es también irreducible en $\Z[\sqrt{-6}]$. 

Queda como ejercicio demostrar que $2+\sqrt{-6}$ y $2-\sqrt{-6}$ son irreducibles,.
\end{example}

\begin{exercise}
Demuestre que $\Z[\sqrt{-5}]$ no es un dominio de factorización única.
\end{exercise}

% 6=2\cdot 3=(1+\sqrt{-5})(1-\sqrt{-5})

Terminamos el capítulo con una aplicación a la teoría de números.

\begin{theorem}[Fermat]
\index{Teorema!de Fermat}
Para $p$ un número primo, son equivalentes:
\begin{enumerate}
	\item $p=2$ o bien $p\equiv1\bmod 4$.
	\item Existe $a\in\Z$ tal que $a^2\equiv-1\bmod p$.
	\item $p$ no es irreducible en $\Z[i]$.
	\item $p$ es suma de dos cuadrados.
\end{enumerate}	
\end{theorem}

\begin{proof}
Veamos primero que $(1)\implies (2)$. Si $p=2$, entonces $a=1$. Si $p=4k+1$, el pequeño teorema de Fermat nos dice que 
las raíces del polinomio $X^{p-1}-1$ 
con coeficientes en $\Z/p$ son $1,2,\dots,p-1$. Escribimos
\[
X^{p-1}-1=X^{4k}-1=(X^{2k}-1)(X^{2k}+1)=(X-1)(X-2)\cdots (X-(p-1))
\]
en $(\Z/p)[X]$. Como $p$ es primo, $\Z/p$ es un cuerpo y luego $(\Z/p)[X]$ es un dominio de factorización única, pues
$(\Z/p)[X]$ es un dominio de ideales principales por ser un dominio euclidiano. Existe entonces $\alpha\in\Z/p$
tal que $\alpha^{2k}+1=0$ y para terminar la demostración alcanza con tomar $a=\alpha^{2k}$. 

Demostremos ahora que $(2)\implies(3)$. Si $a^2\equiv -1\bmod p$, entonces $a^2+1=kp$ para algún $k\in\Z$. Como
$(a-i)(a+i)=a^2+1=kp$, entonces $p\mid (a-i)(a+i)$. Afirmamos que $p\nmid a+i$ en $\Z[i]$. Si $p\mid a+i$, entonces
$a+i=p(e+fi)$ para ciertos $e,f\in\Z$. Luego $1=pf$, una contradicción. De la misma forma se demuestra que $p\nmid a-i$. 
Sabemos entonces que $p\mid (a-i)(a+i)$ pero $p\nmid a-i$ y $p\nmid a+i$, es decir $p$ no es un primo de $\Z[i]$, por lo que 
tampoco será un irreducible de $\Z[i]$. 

Veamos ahora que $(3)\implies(4)$. Si $p=(a+bi)(c+di)$ con $a+bi\not\in\mathcal{U}(\Z[i])$ y $c+di\not\in\mathcal{U}(\Z[i])$, 
es decir $N(a+bi)\ne 1$ y $N(c+di)\ne 1$, entonces
\[
p^2=N(p)=N(a+bi)N(c+di)=(a^2+b^2)(c^2+d^2).
\]
Como $\Z$ es un dominio de factorización única y $p$ es irreducible en $\Z$, $p=a^2+b^2$. 

Para finalizar, demostremos que $(4)\implies(1)$. Como $p$ es un número primo, entonces $p=2$, $p\equiv 1\bmod 4$ o bien $p\equiv 3\bmod 4$. Si $p\equiv 3\bmod 4$ y $p=a^2+b^2$, entonces
$a^2+b^2\equiv 3\bmod 4$, una contradicción pues los únicos casos posibles son $a^2+b^2\equiv 0\bmod 4$, $a^2+b^2\equiv 1\bmod 4$ o bien $a^2+b^2\equiv 2\bmod 4$.     
\end{proof}

%Veamos otra aplicación del anillo $\Z[i]$. Primero, un resultado auxiliar.
%
%\begin{lemma}
%	Si $a,b\in\Z\setminus\{0\}$ son coprimos y $m\in\Z$, entonces $a+bi$ divide a $m$ en $\Z[i]$ si y sólo si $N(a+bi)$ divide a $m$ en $\Z$.
%\end{lemma}
%
%\begin{proof}
%	Supongamos que $N(a+bi)\mid m$ en $\Z$, es decir $m=N(a+bi)k$ para algún $k$. Entonces $a+bi\mid m$ en $\Z[i]$, pues 
%	$m=(a^2+b^2)k=(a+bi)(a-bi)k$.
%	
%	Supongamos ahora que $a+bi$ divide a $m$, es decir 
%	\[
%	m=(a+bi)(c+di)=(ac-bd)+i(bc+ad)
%	\]
%	para ciertos $c,d\in\Z$. Como entonces $bc+ad=0$, tenemos $ad=-bc$, de donde se obtiene que $a\mid bc$ y $b\mid ad$. Como $a$ y $b$ son coprimos,
%	$a\mid c$ y $b\mid d$, es decir $c=ax$ y $d=by$ para ciertos $x,y\in\Z$. Como entonces
%	$a(by)=ad=-bc=-b(ax)$, entonces $ab(y+x)=0$ y luego $x=-y$. En conclusión, 
%	\[
%	m=(a+bi)(c+di)=a^2x-b^2y+i(abx+aby)=x(a^2+b^2)=xN(a+bi).\qedhere
%	\]
%\end{proof}
	
Veamos ahora como aplicación que $\Z[i]$ puede utilizarse para resolver ecuaciones en $\Z$. 	
	
\begin{proposition}
La ecuación $y^3-1=x^2$ tiene únicamente una solución en $\Z$.
\end{proposition}

\begin{proof} 
Si $x$ es impar, entonces $x^2\equiv 1\bmod 4$. Luego $2\mid x^2+1$, pero además $4\nmid x^2+1$. Como además $y$ es par, tenemos
que $y^3=x^2+1$ es divisible por $8$, una contradicción. Luego $x$ es par e $y$ es impar. Escribimos
\[
y^3=x^2+1=(x-i)(x+i)
\]
Observemos que $x-i$ y $x+i$ no tienen factores en común pues si $d\in\Z[i]$ es tal que
$d\mid x+i$ y $d\mid x-i$, entonces $d=1$ por lo que vimos en el ejemplo~\ref{exa:Z[i]div} de la página~\pageref{exa:Z[i]div}. La factorización
única del anillo $\Z[i]$ implica que $x+i=((a+bi)u)^3$ para ciertos $a,b\in\Z$ y $u\in\mathcal{U}(\Z[i])=\{1,-1,i,-i\}$. 
Como entonces $u^3\in\{-1,1,i,-i\}$, sin perder generalidad podemos suponer que 
\[
x+i=(a+bi)^3=(a^3-3ab^2)+i(3a^2b-b^3).
\]
En particular, $1=3a^2b-b^3=b(3a^2-b^2)$, lo que implica que $b=1$ y $a=0$, es decir $(x,y)=(0,1)$. 
\end{proof}
