\chapter{Polinomios}

\index{Polinomio!en una variable}
Sea $R$ un anillo conmutativo. 
Un polinomio con coeficientes en $R$ es una expresión
de la forma
\[
	f=a_nX^n+a_{n-1}X^{n-1}+\cdots+a_1X+a_0=\sum_{i=0}^n a_nX^i,
\]
donde $n\in\N_0$. El conjunto de polinomios en una variable 
con coeficientes en $R$ será denotado por $R[X]$. 

Diremos que $f$ y $g=b_nX^n+b_{n-1}X^{n-1}+\cdots+b_1X+b_0$ son iguales si y
sólo si $ a_i=b_i$ para todo $i\in\{0,1,\dots,n\}$. 

\index{Grado!de un polinomio}
\index{Polinomio!constante}
El grado de un polinomio $f=\sum_{i=0}^na_iX^i$ se define como el menor entero positivo tal que $a_n\ne0$. Un polinomio de grado cero
será denominado \textbf{polinomio constante}.

\index{Polinomio!mónico}
\index{Coeficiente principal!de un polinomio}
Un polinomio $f=\sum_{i=0}^na_iX^i$ de grado $n$ se dice \textbf{mónico} si su \textbf{coeficiente principal} es igual a uno, es decir $a_n=1$.   
Si 
\[
f=\sum a_iX^i,\quad
g=\sum b_jX^j,
\]
donde omitimos los índices de la sumatoria para aliviar un poco la notación, se definen la
suma y el producto como
\[
f+g=\sum (a_k+b_k)X^k,
gh=\sum\sum a_ib_jX^{i+j}.
\]
Con estas operaciones, $R[X]$ es un anillo conmutativo. Además $R$ puede pensarse como un subanillo de $R[X]$ pues
existe un morfismo inyectivo de anillos $R\to R[X]$.  

\begin{example}
Si $f$ es un polinomio mónico y $g$ es un polinomio, existen entonces únicos $q\in R[X]$ y $r\in R[X]$ tales que
\[
g=fq+r,
\]
donde $r=0$ o bien $\deg r<\deg f$. Por ejemplo, si
\[
f=X^5+X^4-3X^3+4X^2+2X,\quad
g=X^4+3X^3-X^2-6X-2
\]
entonces $q=X-2$ y $r=4X^3+8X^2-8X-4$ pues 
\[
f=(X-2)g+(4X^3+8X^2-8X-4).
\]
\end{example}

El algoritmo de división podrá hacerse siempre que el coeficiente principal de $f$ sea una unidad del anillo $R$. En particular, siempre podremos
utilizar el algoritmo de división cuando el anillo $R$ es en realidad un cuerpo y $f\ne 0$.  

\begin{proposition}
Si $g\in R[X]$ y $\alpha\in R$, el resto de dividir al polinomio $g$ por $X-\alpha$ es $g(\alpha)$. En particular, $X-\alpha$ divide al polinomio  
$g$ en $R[X]$ si y sólo si $g(\alpha)=0$. 
\end{proposition}

\begin{proof}
Hay que evaluar en $\alpha$ la expresión $g=(X-\alpha)q+r$. 	
\end{proof}

Como corolario puede demostrarse que 
todo $f\in R[X]$ no nulo tiene a lo sumo $\deg f$ raíces en $R$. 

\begin{proposition}
\label{pro:polinomios}
Sea $\varphi\colon R\to S$ un morfismo de anillos y sea $\alpha\in S$. Existe un único morfismo de anillos $\Phi\colon R[X]\to S$ 
tal que $\Phi(X)=\alpha$ y tal que $\Phi$ coincide con $\varphi$ en los polinomios constantes.  	
\end{proposition}

\begin{proof}[Bosquejo de la demostración]
Si $f=\sum a_iX^i$, definimos 
\[
\Phi(f)=\Phi(\sum a_iX^i)=\sum\Phi(a_i)\Phi(X)^i=\sum\Phi(a_i)\alpha^i,
\] 	
y así $\Phi$ quedaría unívocamente determinado. Dejamos como ejercicio demostrar que $\Phi$ es un morfismo de anillos. 
%Veamos que $\Phi$ es un morfismo de anillos:
%\begin{align*}
%\Phi&\left((\sum a_iXî)(\sum b_jX^j)\right)
%=\Phi\left(\sum\sum a_ib_jX^{i+j}\right)
%=\sum\sum\Phi(a_ib_jX^{i+j})\\
%&=\sum\sum\Phi(a_i)\Phi(b_j)\alpha^{i+j}
%=\left(\sum\Phi(a_i)\alpha^i\right)\left(\sum\Phi(b_j)\alpha^j\right)
%=\Phi(f)\Phi(g).   	
%\end{align*}
\end{proof}

\begin{example}
Sea $\varphi\colon\R[X]\to\R$, $X\mapsto 2$. Entonces $\ker\varphi$ es el ideal   
$(X-2)$ de $\R[X]$ generado por $X-2$ pues 
\[
\ker\varphi=\{g\in\R[X]:g(2)=0\}
=\{g\in\R[X]:g=(X-2)q\text{ para algún $q\in\R[X]$}\}.
\] 	
\end{example}

El resultado que sigue es análogo al teorema~\ref{thm:Z}.

\begin{theorem}
Sea $K$ un cuerpo. Todo ideal de $K[X]$ es principal. Más aún, todo ideal $I$ no nulo de $K[X]$
está generado por el único polinomio mónico de menor grado que está contenido en $I$.  	
\end{theorem}

\begin{proof}
	Sea $I$ un ideal de $K[X]$. Si $I=\{0\}$, entonces $I$ es principal. Supongamos que $I\ne\{0\}$ 
	y sea $f\in I\setminus\{0\}$ de grado mínimo. Sin perder generalidad, podemos suponer que $f$ es mónico pues
	si no lo fuera, digamos $f=a_nX^n+\cdots$ con $a_n\ne 0$, entonces solamente hay que reemplaar a $f$ 
	por el polinomio $a_n^{-1}f$. 
	
	Veamos que $I=(f)$. Vamos a demostrar la inclusión no trivial. Sea $g\in I$. Escribimos
	$g=fq+r$ para ciertos $q,r\in K[X]$, donde $r=0$ o bien $\deg r<\deg f$. Si $r\ne 0$, entonces $r=g-fq\in I$, 
	una contradicción a la minimalidad del grado de $f$. Luego $r=0$ y entonces $f$ divide a $g$, es decir
	$g\in (f)$.   
\end{proof}

Diremos que un polionomio $f\in\Z[X]$ se dice \textbf{irreducible} si $f$ no es constante 
y $f=gh$ implica que $g$ o $h$ son constantes. 
De la misma forma podemos definir polinomios irreducibles 
en $\Q[X]$. 

Los polinomios de grado uno son irreducibles. 

\begin{example}
\label{exa:traslacion}, 
    Sea $f=\sum_{i=0}^na_iX^i\in K[X]$ y sea $a\in K$. Entonces $f$ es 
    irreducible si y sólo si $f(X+a)=\sum_{i=0}^na_i(X+a)^i$ es irreducible, pues
    \[
    T_a\colon K[X]\to K[X],g\mapsto g(X+a),
    \]
    es isomorfismo de anillos y manda irreducibles en irreducibles. 
\end{example}

Un polinomio $f\in\Z[X]$ se dice \textbf{primitivo} si 
el máximo común divisor de sus coeficientes es igual a uno. 

\begin{lemma}[Gauss]
\index{Lema!de Gauss}
Sean $f,g\in\Z[X]$. Si $f$ y $g$ son primitivos, entonces $fg$ es primitivo.
\end{lemma}

\begin{proof}
Sean $f=\sum_{i=0}^na_iX^i$, $g=\sum_{i=0}^mb_iX^i$. Supongamos que $fg$ no es primitivo y sea $p$ un primo que
divide a todos los coeficientes de $fg$. Como $f$ y $g$ son primitivos, $p$ no 
divide a los coeficientes de $f$ ni divide a los coeficientes de $g$. Sean $i\in\{0,\dots,n\}$ 
y $j\in\{0,\dots,m\}$ minimales tales que $p\nmid a_i$ y $p\nmid b_j$. Si $c_{i+j}$ es el coeficiente de $X^{i+j}$ en $fg$, 
entonces
\[
c_{i+j}=\sum_{k>i}a_kb_{i+j-k}+\sum_{k<i}a_kb_{i+j-k}+a_ib_j.
\]
Luego $p$ divide a $\sum_{k>i}a_kb_{i+j-k}+\sum_{k<i}a_kb_{i+j-k}$ pero no divide al 
entero $a_ib_j$, es decir no divide a $c_{i+j}$, una contradicción.
\end{proof}

\begin{theorem}[Gauss]
\index{Teorema!de Gauss}
\index{Criterio!de irreducibilidad de Gauss}
Sea $f\in\Z[X]$ un polinomio no constante y primitivo. Entonces $f$ es 
irreducible en $\Z[X]$ si y sólo si $f$ es irreducible en $\Q[X]$. 
\end{theorem}

\begin{proof}
Demostremos la implicación no trivial. Vamos a demostrar que $f$ es reducible en 
$\Z[X]$. Supongamos que $f=gh$ con $g,h\in\Q[X]$ de grado positivo. Al multiplicar 
por un número racional adecuado, podemos suponer que
\[
f=\frac{a}{b}g_1h_1,
\]
donde $\gcd(a,b)=1$ y $g_1,h_1\in\Z[X]$ son polinomios primitivos, es decir $bf=ag_1h_1$. Luego
el máximo común divisor de los coeficientes de $bf$ es $b$. Como $g_1h_1$ es primitivo por el lema de Gauss, 
el máximo común divisor de los coeficientes de $ag_1h_1$ es $a$. Luego $a=b$ o $a=-b$, 
es decir $f=g_1h_1$ o $f=-g_1h_1$ en $\Z[X]$. 
\end{proof}

\begin{theorem}[Eisenstein]
\index{Teorema!de Eisenstein}
\index{Criterio!de irreducibilidad de Eisenstein}
Sea $f=\sum_{i=0}^na_iX^i\in\Z[X]$ primitivo de grado $n$. Si existe un primo $p$ tal que
$p\mid a_j$ para $j\in\{0,\dots,n-1\}$, $p\nmid a_n$ y $p^2\nmid a_0$, entonces
$f$ es irreducible en $\Z[X]$.
\end{theorem}

\begin{proof}
Escribimos $f=gh$ con $g=\sum_{i=0}^kb_iX^i$ y $h=\sum_{i=0}^lc_iX^i$, donde suponemos que $b_k\ne 0$ y $c_l\ne 0$. 
Como $p\mid a_0$, $p^2\nmid a_0$ y $a_0=b_0c_0$, entonces $p$ divide a $b_0$ o $p$ divide a $c_0$, pero no a ambos. Supongamos que
$p\mid b_0$ y que $p\nmid c_0$. Como $n=k+l$ y $p\nmid a_n=b_kc_l$, entonces $p\nmid b_k$. Sea $j\in\{1,\dots,k\}$ maximal tal que
$p\mid b_i$ para todo $i<j$ y $p\nmid b_j$. Entonces
\[
a_j=b_jc_0+\underbrace{b_{j-1}c_1+\cdots+b_0c_j}_{\text{divisible por $p$}}
\]
se concluye que $p\nmid b_jc_0$ y luego $p\nmid a_j$. En conclusión, $j=n$ 
y entonces $k=n$ y $h$ es constante.
\end{proof}

\begin{example}
    El polinomio $X^5+16X+2\in\Z[X]$ es irreducible.
\end{example}

\begin{example}
Sea $p$ un número primo. Veamos que $f=X^{p-1}+\cdots+X+1$ es irreducible en $\Q[X]$. 
Escribimos $(X-1)f=X^p-1$ y aplicamos el isomorfismo de anillos 
$T_1\colon g\mapsto g(X+1)$ que vimos en el ejemplo~\ref{exa:traslacion}, 
\[
XT_1(f)=(X+1)^p-1=\sum_{i=1}^p\binom{p}{i}X^i
\]
y luego $T_1(f)=\sum_{i=1}^p\binom{p}{i}X^{i-1}$. Como $p\mid\binom{p}{i}$ para todo $i\in\{1,\dots,p-1\}$ y 
además $p^2\nmid \binom{p}{1}=p$, entonces el criterio de Eisenstein implica que $T_1(f)$ es irreducible. 
Luego $f$ es también irreducible. 
\end{example}

\index{Monomio}
Nos interesará también estudiar anillos de polinomios en varias variables. Un \textbf{monomio} 
en las variables $X_1,\dots,X_n$ es un producto de la forma
\[
X_1^{i_1}X_2^{i_2}\cdots X_n^{i_n}
\]
donde $i_1,\dots,i_n\geq0$. A veces conviene utilizar la siguiente notación: si $i=(i_1,\dots,i_n)$, entonces 
\[
X^i=X_1^{i_1}X_2^{i_2}\cdots X_n^{i_n}.
\]  
El monomio $X^0$, donde $0=(0,0,\dots,0)$, será denotado por $1$. 

Un polinomio en las variables $X_1,\dots,X_n$ 
con coeficientes en $R$ es una combinación lineal (sobre $R$) de finitos 
monomios, es decir
\[
f=f(X_1,\dots,X_n)=\sum a_iX^i,
\]
donde la suma se recorre sobre todos los multi-índices $i=(i_1,\dots,i_n)$, los coeficientes $a_i$ son elementos del anillo $R$
y solamente finitos de esos coeficientes son distintos de cero. El conjunto de los polinomios en las variables
$X_1,\dots,X_n$ con coeficientes en $R$ será denotado por $R[X_1,\dots,X_n]$. 

\begin{exercise}
Demuestre que 
la proposición~\ref{pro:polinomios} vale también en el caso de polinomios en varias variables.
\end{exercise} 

%\index{Polinomio!homogéneo}
%Un polinomio donde todos sus monomios con coeficiente no nulo son de grado $d$ se denomina \textbf{homogéneo}

\begin{example}
Si $R$ es un anillo, $R[X]$ es un anillo. Podemos considerar entonces el anillo de polinomios $R[X,Y]=(R[X])(Y)=R[X][Y]$. 
Sabemos que si utilizamos las identificaciones pertinentes, podemos pensar que $R\subseteq R[X]\subseteq (R[X])[Y]$ son subanillos. Si 
	$\varphi\colon R\to R[X][Y]$ la inclusión, existe entonces un único morfismo de anillos $\Psi\colon R[X,Y]\to R[X][Y]$ que extiende a $\varphi$ y tal que
	$\Psi(X)=X$ y $\Phi(Y)=Y$. Puede demostrarse que $\Psi$ es biyectivo y luego, en consecuencia,
	\[
R[X,Y]\simeq R[X][Y].
\]
\end{example}

\begin{example}
Veamos que $\R[X,Y]/(X)\simeq\R[Y]$. Consideramos el morfismo sobreyectivo $\varphi\colon\R[X,Y]\to\R[Y]$, $\varphi(f(X,Y))=f(0,Y)$. Afirmamos
que $\ker\varphi=(X)$. En efecto, para probar que $\ker\varphi\subseteq (X)$ observamos que si
\[
f(X,Y)=f_0(Y)+f_1(Y)X+\cdots+f_n(Y)X^n,
\]
donde $f_i(Y)\in\R[Y]$ para todo $i\in\{0,1,\dots,n\}$, entonces 
\[
0=\varphi(f(X,Y))=f(0,Y)=f_0(Y).
\] 
Luego $f(X,Y)=f_1(Y)X+\cdots+f_n(Y)X^n=X(f_1(Y)+\cdots+f_n(Y)X^{n-1})\in (X)$. La otra inclusión es trivial. 
El primer teorema de isomorfismos, entonces, implica que $\R[X,Y]/(X)\simeq\R[Y]$.  
\end{example}

\begin{exercise}
Demuestre que $\R[X]/(X^2-1)\simeq\R\times\R$.	
\end{exercise}

\begin{exercise}
Demuestre que $\Q[X]/(X-2)\simeq\Q$. 	
\end{exercise}
	
