\chapter{Grupos simples}

\index{Grupo!simple}
Recordemos que un grupo $G$ es \textbf{simple} si $G\ne\{1\}$ y sus únicos subgrupos normales
son $G$ y $\{1\}$.

\begin{example}
Si $p$ es un número primo, $\Z/p$ es simple.
\end{example}

\index{Clase de conjugación}
No es difícil demostrar que $\Alt_5$ es simple, pero necesitamos repasar algunos conceptos. Si $G$ es un grupo y $g\in G$, la clase de conjugación de $g$ en $G$ es el conjunto
$\{xgx^{-1}:x\in G\}$. Una observación sencilla pero importante: Si el subgrupo $N$ de $G$ es normal en $G$, entonces $N$ es uníón de clases de conjugación de $G$, y una de esas clases es $\{1\}$, pues 
\[
N=\bigcup_{n\in N}\{xnx^{-1}:x\in G\}.
\]  

%\begin{proposition}
%El grupo $\SL_3(2)$ es simple. 
%\end{proposition}

%\begin{proof}
%El grupo $\SL_3(2)=\{a\in\GL_3(2):\det(a)=1\}$ tiene orden... 
%\end{proof}

%Veamos otro ejemplo:
Veamos una aplicación sencilla de la afirmación anterior. 

\begin{proposition}
El grupo alternado $\Alt_5$ es simple.
\end{proposition}

\begin{proof}
Para demostrar el teorema vamos a contar los tamaños de las clases de conjugación de $\Alt_5$. Las clases de conjugación de $\Alt_5$ y sus tamaños son:
\begin{align*}
\id && 1\\
(123) && 20\\
(12)(34) && 15\\
(12345) && 12\\
(21345) && 12
\end{align*}
Si $N$ es un subgrupo normal de $\Alt_5$, entonces $N$ es unión de clases de conjugación de $\Alt_5$ y una de esas clases es $\{id\}$. Sin embargo, ninguna unión de clases de conjugación
de $\Alt_5$ que incluya $\{id\}$ tendrá tamaño un divisor de 60, a menos que $N=\{\id\}$ o bien que  $N=\Alt_5$.  
\end{proof} 

Nuestro objetivo es demostrar que los grupos alternados $\Alt_n$ son simples siempre que $n\geq5$. Para eso, necesitamos demostrar
varios lemas auxiliares. 

\index{Estructura cíclica}
Recordemos toda permutación $\rho\in\Sym_n$ puede descomponerse como producto de ciclos disjuntos, digamos
\[
\rho=(a_1\cdots a_r)(b_1\cdots b_s)\cdots (c_1\cdots c_t)
\]
donde por convención omitiremos aquellos ciclos de longitud uno. 
La estrucutra cíclica de $\rho$ será entonces la sucesión ordenada de los números $r,s,\dots t$, 
donde convenientemente omitiremos los puntos fijos. Por ejemplo, la estructura cíclica de la trasposición $(ab)$ es 2, 
del 3-ciclo $(abc)(d)$ es 3 y de la permutación $(123)(45)(789a)(bcd)(d)$ es 2,3,3,4. 
  
El primer lema que demostraremos afirma que dos permutaciones tienen la misma estructura cíclica si y sólo si son conjugadas. 

\begin{lemma}
Si $\rho_1$ y $\rho_2$ son permutaciones de $\Sym_n$ con la misma estructura cíclica, entonces
$\rho_2=\sigma\rho_1\sigma^{-1}$ para alguna permutación $\sigma\in\Sym_n$. 
\end{lemma}

\begin{proof}
Supongamos que
\begin{align*}
\rho_1=(a_1\cdots a_r)(b_1\cdots b_s)\cdots (c_1\cdots c_t), && 
\rho_2=(x_1\cdots x_r)(y_1\cdots y_s)\cdots (z_1\cdots z_t).
\end{align*}
Sean 
\begin{align*}
\Fix(\rho_1) &= \{x\in\{1,\dots,n\}:\rho_1(x)=x\}=\{k_1,\dots,k_m\},&&
\Fix(\rho_2)=\{l_1,\dots,l_m\}	
\end{align*}
los puntos fijos de las permutaciones $\rho_1$ y $\rho_2$, respectivamente. Entonces
\[
\sigma(x)=\begin{cases}
x_j & \text{si $x=a_j$ para algún $j$},\\
y_j & \text{si $x=b_j$ para algún $j$},\\
\vdots\\
z_j & \text{si $x=c_j$ para algún $j$},\\
l_j & \text{si $x=k_j$ para algún $j$},	
\end{cases}
\]
cumple que $\sigma\rho_1\sigma^{-1}=\rho_2$. 
\end{proof}

El siguiente lema es la variante del anterior correspondiente al grupo alternado. 
Nos interesa saber cuándo dos permutaciones conjugadas en $\Sym_n$ son también conjugadas en $\Alt_n$.

\begin{lemma}
Si $\rho_1,\rho_2\in\Sym_n$ son conjugados en $\Sym_n$ y además $|\Fix(\rho_1)|\geq2$, entonces
$\mu\rho_1\mu^{-1}=\rho_2$ para algún $\mu\in\Alt_n$.  
\end{lemma}

\begin{proof}
Supongamos que $\rho_2=\sigma\rho_1\sigma^{-1}$ para algún $\sigma\in\Sym_n$. 
Por hipótesis, sabemos que existen $a,b\in\{1,\dots,n\}$ tales que $\rho_1(a)=a$, $\rho_1(b)=b$ y $a\ne b$. Sea
\[
\mu=\begin{cases}
\sigma & \text{si $\sigma\in\Alt_n$,}\\
\sigma(ab) & \text{en caso contrario.}
\end{cases}
\]
Entonces $\mu\in\Alt_n$ y además $\mu\rho_1\mu^{-1}=\rho_2$ pues $(ab)$ conmuta on $\rho_1$. 
\end{proof}

Veamos algunos ejemplos.

\begin{example}
Si $\rho_1=(23)(156)$ y $\rho_2=(45)(123)$, entonces 
el lema anterior nos dice que $\rho_2=\sigma\rho_1\sigma^{-1}$ si 
\[
\sigma=\binom{123456}{145623}.
\]
\end{example}

\begin{example}
Los 3-ciclos $\rho_1=(123)$ y $\rho_2=(132)$ son conjugados en $\Sym_3$ pues 
$(123)=\sigma(132)\sigma^{-1}$ si $\sigma=(23)$. Sin embargo, $\rho_1$ y $\rho_2$ 
no son conjugados en $\Alt_3$. 
\end{example}

Estamos en condiciones de probar el teorema del capítulo.

\begin{theorem}[Jordan]
\index{Teorema!de Jordan}
\index{Simplicidad!de $\Alt_n$}
Si $n\geq5$, $\Alt_n$ es simple. 
\end{theorem}

\begin{proof}
Sea $N\ne\{\id\}$ un subgrupo normal de $\Alt_n$. Si $(abc)\in N$, entonces cualquier 3-ciclo también está en $N$ (pues todos los 3-ciclos son conjugados en $\Sym_n$ y por el lema anterior
sabemos que $(ijk)=\mu(abc)\mu^{-1}\in N$ para algún $\mu\in\Alt_n$. Luego $N=\Alt_n$. 

Vamos a demostrar ahora que nuestro $N$ siempre contiene un 3-ciclo. Como $N$ es no trivial, existe $\sigma\in N\setminus\{\id\}$. Sean $m=|\sigma|$ y $p$ un primo tal que divide a $m$. 
Entonces $\tau=\sigma^{m/p}$ tiene orden $p$ y luego $\tau=\rho_1\cdots\rho_s$, donde los $\rho_j$ son $p$-ciclos disjuntos. 

Si $p=2$, entonces $1=\sgn(\tau)=(-1)^s$ y luego $s$ es par. Escribimos 
\[
\tau=(ab)(cd)\rho_3\cdots\rho_s
\]
y entonces, como $\rho_3\cdots\rho_s$ conmuta con $(abc)$ y $(acb)$, 
\[
(ac)(bd)\tau=(abc)\tau(abc)^{-1}\in N
\]
y luego $(ab)(cd)\in N$. Sea $e\in\{1,\dots,n\}\setminus\{a,b,c,d\}$. Entonces
\[
(ae)(bd)=(aec)\underbrace{(ac)(bd)}_{\in N}(aec)^{-1}\in N
\]
y luego 
\[
(aec)=(ac)(ae)=(ac)(bd)(ae)(bd)\in N.
\]

Si $p=3$, podemos suponer sin perder generalidad que $s\geq2$ (pues de lo contrario $\tau$ sería un 3-ciclo). Entonces
$\tau=(abc)(def)\rho_3\cdots\rho_s$. Como $(bcd)$ conmuta con $\rho_3\cdots\rho_s$ y $N$ es normal en $\Alt_n$, entonces
\begin{align*}
&(adbce)=(bcd)\tau(bcd)^{-1}\tau^{-1}\in N
\shortintertext{y luego}
&(adc)=(adb)(adbce)(adb)^{-1}(adbce)^{-1}\in N.
\end{align*}

Si $p>3$, entonces $\tau=(abcd\cdots z)\rho_2\cdots\rho_s$. En particular, $(abc)$ conmuta con $\rho_2\cdots\rho_s$ y entonces
\[
(abd)=(abc)\tau(abc)^{-1}\tau^{-1}\in N.\qedhere
\]
\end{proof}

Como aplicación, vamos a calcular los subgrupos normales de $\Sym_n$ para $n\geq5$. 

\begin{proposition}
\index{Subgrupos normales!de $\Sym_n$}
Sea $n\geq5$ y sea $N$ un subgrupo normal de $\Sym_n$. Entonces $N=\{\id\}$, $N=\Alt_n$ o bien $N=\Sym_n$. 
\end{proposition}

\begin{proof}
Sea $N$ un subgrupo normal de $\Sym_n$. Como la restricción $\sgn|_N$ de la función signo a $N$ es un morfismo de grupos, 
\[
(N:\ker(\sgn|_N))=|\sgn(N)|\in\{1,2\}.
\] 
Si $(N:\ker(\sgn|_N)=1$, entonces $N\subseteq\Alt_n$ y entonces $N$ es normal en $\Alt_n$. 
Luego $N=\{\id\}$ o bien $N=\Alt_n$ pues $\Alt_n$ es un grupo simple si $n\geq5$. 

Si $(N:\ker(\sgn|_N)=2$, entonces $N\cap\Alt_n=\ker(\sgn|_N)$ es normal en $\Alt_n$ y luego, por la simplicidad de $\Alt_n$ para $n\geq5$, $N\cap\Alt_n=\{\id\}$ o bien $N\cap\Alt_n=\Alt_n$. 

En el primer caso, $|N|=2$ y entonces $N$ contiene una permutación impar que además tiene orden dos, digamos
$\tau=(ij)\tau_2\cdots\tau_s$, escrita como producto de trasposiciones disjuntas. Entonces
$\pi=(ik)\tau(ik)\in N$ si $k\not\in\{i,j\}$ y $\pi\ne \tau$ pues $\tau(j)=i$ y $\pi(j)=k$. Luego $|N|\geq3$, una contradicción. 

En el segundo caso, si $N\cap\Alt_n=\Alt_n$, entonces $N=\Alt_n$.
\end{proof}
