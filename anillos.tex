\chapter{Anillos}

En esta parte trataremos sobre algunas propiedades básicas de ciertas
estructuras que se conocen como anillos conmutativos. Si bien los anillos no conmutativos tienen gran importancia dentro de la matemática, 
las aplicaciones que daremos en este curso, estarán basadas en la teoría de anillos conmutativos.

\begin{definition}
Un anillo es un conjunto $R$ con dos operaciones binarias, una suma $(x,y)\mapsto x+y$ y un producto $(x,y)\mapsto xy$ de forma tal que  
se cumplen las siguientes propiedades:
\begin{enumerate}
\item $(R,+)$ es un grupo abeliano (escrito aditivamente).
\item $(xy)z=x(yz)$ para todo $x,y,z\in R$.
\item Existe $e\in R$ tal que $xe=ex=x$ para todo $x\in R$.
\item $x(y+z)=xy+xz$ para todo $x,y,z\in R$.
\item $(x+y)z=xz+yz$ para todo $x,y,z\in R$.
\end{enumerate}
\end{definition}

\begin{definition}
Un anillo $R$ se dirá \textbf{conmutativo} si $xy=yx$ para todo $x,y\in R$.
\end{definition}

\begin{examples}
$\N$ no es un anillo. En cambio, 
$\Z$, $\Q$, $\R$ y $\C$ sí son anillos conmutativos.
\end{examples}

\begin{example}
$\Z/n$ es un anillo conmutativo.
\end{example}

\begin{example}
Si $R$ es un anillo, entonces 
\[
R[X]=\left\{\sum_{i=0}^n a_iX^i:n\in\N_0,\,a_0,\dots,a_n\in R\right\}
\]
es un anillo con las operaciones usuales.    	
\end{example}

Como dijimos al principio del capítulo, los anillos no conmutativos tienen gran importancia dentro de la matemática. Para convencernos, 
tenemos el siguiente ejemplo. Si $n\geq2$, entonces 
$M_n(\R)=\R^{n\times n}$ es un anillo no conmutativo con las operaciones usuales. 

\begin{exercise}
Si $A$ es un grupo abeliano, entonces $\End(A)$ es un anillo conmutativo con 
las operaciones
\[
(f+g)(x)=f(x)+g(x),\quad
(fg)(x)=f(g(x)).
\]	
\end{exercise}

Si $R$ es un anillo, es fácil verificar las siguientes propiedades:
\begin{enumerate}
	\item $x0=0x=0$ para todo $x\in R$.
	\item $x(-y)=-xy$ para todo $x,y\in R$.
	\item Si $1=0$, entonces $|R|=1$. Este anillo se conoce como el \textbf{anillo nulo}.
\end{enumerate}

Si $R$ es un anillo y $S$ es un subconjunto de $R$, diremos que 
$S$ es cerrado por multiplicación si $S\cdot S\subseteq S$, 
es decir $st\in S$ si $s,t\in S$. 

\begin{definition}
\index{Subanillo}
Sea $R$ un anillo. 
Un \textbf{subanillo} de $R$ es un subconjunto $S$ de $R$ tal que $(S,+)$ es un subgrupo abeliano de $(R,+)$, $S$ es cerrado por multiplicación y además $1\in S$. 
\end{definition}

\begin{examples}\
\begin{enumerate}
\item $\Z\subseteq \Q\subseteq\R\subseteq\C$ son subanillos. 
\item $\Z$ es un subanillo de $\Z$.
\item $\Z[i]=\{a+bi:a,b\in\Z\}$ es un subanillo de $\C$. 
\item $\Q[\sqrt{2}]=\{a+b\sqrt{2}:a,b\in\Q\}$ es un subanillo de $\R$.
\end{enumerate}
\end{examples}

\begin{example}
\index{Centro!de un anillo}
Si $R$ es un anillo, $Z(R)=\{x\in R:xy=yx\text{ para todo $y\in R$}\}$ es un subanillo de $R$. Se denomina el \textbf{centro} de $R$.
\end{example}

\begin{exercise}
Si $S$ es un subanillo de $R$, entonces $0_S=0_R$. 
Además si $x\in S$, el inverso aditivo de $x$ en $S$ coincide con el inverso aditivo de $x$ en $R$.	
\end{exercise}

\begin{exercise}
Si $S$ y $T$ son subanillos de $R$, entonces $S\cap T$ es también un subanillo de $R$.	
\end{exercise}

El resultado del ejercicio anterior puede generalizarse a una intersección arbitraria de subanillos. 

\begin{exercise}
Si $R_1\subseteq R_2\subseteq\cdots$ es una sucesión de subanillos de un anillo $R$, entonces $\cup_{i\geq1}R_i$ es un subanillo de $R$.	
\end{exercise}

\begin{definition}
\index{Unidad!de un anillo}
Sea $R$ es un anillo. Un elemento $x\in R$ es una \textbf{unidad} si existe $y\in R$ tal que $xy=yx=1$.   	
\end{definition}

Si un elemento $x$ es una unidad, entonces el inverso $y$ tal que $xy=yx=1$ es único. Esto nos permite 
escribir $y=x^{-1}$.  
El \textbf{grupo de unidades} de $R$ se define como
\[
\mathcal{U}(R)=\{x\in R:x\text{ es una unidad}\}.
\]

En anillos no conmutativos conviene distinguir unidades, unidades a derecha y unidades a izquierda, ya que todos
estos conceptos son diferentes. Veamos un ejemplo. Sea $R=\End(V)$, donde $V$ es un espacio vectorial
con base $e_1,e_2,\dots$. Sea $f\in R$ tal que $f(e_i)=e_{i+1}$ para todo $i$ y sea $g\in R$ tal que
\[
g(e_i)=\begin{cases}
0 & \text{si $i=1$},\\
e_{i-1} & \text{si $i>1$}.	
\end{cases}
\]
Entonces $g\circ f=\id$ pero $f\circ g\ne\id$ pues $f(g(e_1))=f(0)=0$. 
Luego $f$ no es una unidad. En caso contrario, existe $h\in R$ tal que $f\circ h=h\circ f=\id$ y entonces
\[
g=g\circ\id=g\circ (f\circ h)=(g\circ f)\circ h=\id\circ h=h,
\]
una contradicción. 

\begin{definition}
\index{Anillo!de división}
Diremos que un anillo $R$ es de \textbf{división} si $\mathcal{U}(R)=R\setminus\{0\}$.  	
\end{definition}

\begin{example}
Sea 
\[
R=\R+\R i+\R j+\R k=\{a+bi+cj+dk:a,b,c,d\in\R\}
\]
Entonces $R$ es un anillo de división no conmutativo con la suma usual y 
la multiplicación inducida por 
$i^2=j^2=k^2=-1$, $ij=k$, $jk=i$, $ki=j$. 
\end{example}

\begin{definition}
\index{Cuerpo}
Un \textbf{cuerpo} es un anillo de división conmutativo. 	
\end{definition}

\begin{examples}
$\Q$, $\R$ y $\C$ son cuerpos. Si $p$ es un número primo, entonces $\Z/p$ es un cuerpo.	
\end{examples}

\begin{example}
$\Q[\sqrt{2}]=\{x+\sqrt{2}y:x,y\in\Q\}$ es un cuerpo con las operaciones usuales. 	
\end{example}






